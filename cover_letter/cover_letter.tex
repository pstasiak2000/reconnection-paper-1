\documentclass[a4paper,10pt]{letter}
\usepackage[margin=1in]{geometry}
\usepackage{lipsum}
\usepackage{xcolor}
\def\red#1{\textcolor{red}{#1}}


\begin{document}

% Sender Information
\begin{flushright}
    \begin{tabular}{l}
        \textbf{Piotr Stasiak} \\
        School of Mathematics, Statistics and Physics\\
        Newcastle University \\
        Newcastle  upon Tyne \\
        p.stasiak@newcastle.ac.uk \\
        \today
    \end{tabular}
\end{flushright}

\vspace{1cm}

% Recipient Information
The Editors,\\
Physical Review Letters\\
APS\\


\vspace{1cm}

Dear Editor,

\vspace{0.5cm}

We are pleased to submit our manuscript entitled 
\textbf{``Experimental and theoretical evidence of universality 
in superfluid vortex reconnections''} to be
considerated for publication in \textit{Physical Review Letters}. 
This paper presents experimental and numerical insights into the 
dynamics of vortex reconnections in superfluid helium at zero and 
non-zero temperatures, with significant implications 
for understanding the physics of vortex reconnections
with respect to energy dissipation/transfer and irreversibility.

For the first time in the study of vortex reconnections, 
we combine experiments with numerical simulations, finding very good agreement. 
Our work shows that the scaling law describing the minimum distance 
between pre and post reconnecting vortices
%, $\delta_{\pm} = A^\pm(\kappa |t -t_0)^{1/2}$, where $A_{-}$ and $A_{+}$ are the approach 
%and separation prefactors associated with pre- and post-reconnection dynamics across a range 
%of temperatures.
is universal: it is not only valid in Bose-Einstein Condensates (BECs) and 
classical viscous fluids (as established in previous studies), but also in superfluid 
helium. As in BECs and Navier-Stokes fluids, superfluid vortices separate faster than they
approach each other.
This property is related to dissipation of kinetic energy, hence to irreversibility. 
%We hence show that regardless of the 
%small-scale physics triggering the reconnections, a universal behaviour 
%related to irreversibility is observed across fluid systems. 

We also demonstrate that each reconnection event in superfluid helium injects energy 
into the normal fluid. 
%, a process with direct implications for sustaining turbulence in quantum fluids. 
By estimating the reconnection frequency and examining energy dissipation, we show that 
if the vortex line density is sufficiently large, these energy injections 
can maintain the normal fluid in a dynamically perturbed state.

We believe that our findings will be of interest to readers of \textit{Physical Review Letters}, 
as they provide a deeper understanding of superfluid dynamics and highlight universal aspects of 
vortex reconnections across fluid systems. 

Thank you for considering this manuscript; we look forward to your response.
\vspace{0.5cm}

On behalf of the authors,

\vspace{1cm}

\textbf{Piotr Stasiak}

\end{document}
