% ****** Start of file apssamp.tex ******
%
%   This file is part of the APS files in the REVTeX 4.2 distribution.
%   Version 4.2a of REVTeX, December 2014
%
%   Copyright (c) 2014 The American Physical Society.
%
%   See the REVTeX 4 README file for restrictions and more information.
%
% TeX'ing this file requires that you have AMS-LaTeX 2.0 installed
% as well as the rest of the prerequisites for REVTeX 4.2
%
% See the REVTeX 4 README file
% It also requires running BibTeX. The commands are as follows:
%
%  1)  latex apssamp.tex
%  2)  bibtex apssamp
%  3)  latex apssamp.tex
%  4)  latex apssamp.tex
%
\documentclass[%
 reprint,
%superscriptaddress,
%groupedaddress,
%unsortedaddress,
%runinaddress,
%frontmatterverbose, 
%preprint,
%preprintnumbers,
%nofootinbib,
%nobibnotes,
%bibnotes,
 amsmath,amssymb,
 aps,
 prl,
%pra,
% prb,
% rmp,
%prstab,
%prstper,
%floatfix,
]{revtex4-2}

\usepackage{graphicx}% Include figure files
\usepackage{dcolumn}% Align table columns on decimal point
\usepackage{bm}% bold math
\usepackage{blindtext}
\usepackage{float}
\usepackage{caption}
\usepackage{cleveref}
\usepackage{subcaption}
\usepackage{xcolor}
%\usepackage{hyperref}% add hypertext capabilities
%\usepackage[mathlines]{lineno}% Enable numbering of text and display math
%\linenumbers\relax % Commence numbering lines

%\usepackage[showframe,%Uncomment any one of the following lines to test 
%%scale=0.7, marginratio={1:1, 2:3}, ignoreall,% default settings
%%text={7in,10in},centering,
%%margin=1.5in,
%%total={6.5in,8.75in}, top=1.2in, left=0.9in, includefoot,
%%height=10in,a5paper,hmargin={3cm,0.8in},
%]{geometry}


\newcommand{\etal}{{\it et al.}~}
\newcommand{\bom}{\boldsymbol{\omega}}

% \newcommand{\sd}[3][\null]{\ensuremath{\dfrac{\d^{#1} #2}{\d #3^{#1}}}}%Standard derivative 
% \newcommand{\pd}[3][\null]{\ensuremath{\dfrac{\partial^{#1} #2}{\partial #3^{#1}}}}%Partial derivative
% \newcommand{\matd}[2][\null]{\ensuremath{\dfrac{\mathrm{D}^{#1} #2}{\mathrm{D} t^{#1}}}} %Material derivative

\def \s{\mathbf{s}}
\def \v{\mathbf{v}}
\def \x{\mathbf{x}}
\def \r{\mathbf{r}}
\def \k{\mathbf{k}}

\def \cm{\mathrm{cm}}
\def \cms{\mathrm{cm/s}}
\def \sec{\mathrm{s}}
\def \K{\mathrm{K}}

\def\red#1{\textcolor{red}{#1}}
\def\blue#1{\textcolor{blue}{#1}}
\newcommand*{\NOTE}[1]{\textbf{\color{red}[#1]}}
%%%%%%%%%%%%%%%%%%%%%%%%%%%%%%%%%



\begin{document}

\preprint{APS/123-QED}

\title{Superfluid vortex reconnections at non-zero temperatures}

\author{P. Z. Stasiak}
\author{A. Baggaley}
\author{C.F. Barenghi}
\affiliation{School of Mathematics, Statistics and Physics, Newcastle University, Newcastle upon Tyne, NE1 7RU, United Kingdom}

\author{G. Krstulovic}
\affiliation{Universit\'e C\^ote d'Azur, Observatoire de la C\^ote d'Azur, CNRS,Laboratoire Lagrangre, Boulevard de l'Observatoire CS 34229 - F 06304 NICE Cedex 4, France}

\author{L. Galantucci}
\affiliation{Istituto per le Applicazioni del Calcolo ``M. Picone" IAC CNR, Via dei Taurini 19, 00185 Roma, Italy}

\author{\red{W. Guo}}
\author{\red{Y. Xin}}
\author{\red{Y. Alihosseini}}
\affiliation{\red{Department of Mechanical Engineering, FAMU-FSU College of Engineering, Florida State University, Tallahassee, Florida 32310, USA}}
\date{\today}% It is always \today, today,
             %  but any date may be explicitly specified

\begin{abstract}
The minimum separation between reconnecting vortices
in fluids and superfluids obeys a universal scaling law with respect to time.
The pre-reconnection and the post-reconnection prefactors 
of this scaling law are different, a property related to irreversibility.
Using experiments and a numeric model which fully accounts for the independent dynamics of the superfluid
vortex lines and the thermal normal fluid component, we determine the temperature
dependence of these prefactors. We also numerically show that each vortex reconnection event
represents a sudden injection of energy in the normal fluid. Finally we argue that
in a turbulent flow, these punctuated energy injections can sustain the normal fluid
in a perturbed state, provided that the density of superfluid vortices is large enough.
\end{abstract}

%\keywords{Suggested keywords}%Use showkeys class option if keyword
   %display desired
\maketitle

\begin{figure*}[t]
	\centering
	\begin{subfigure}[b]{0.24\textwidth}
		\centering
		\includegraphics*[width=\textwidth]{exp-snap-1.png}
	\end{subfigure}
	\begin{subfigure}[b]{0.24\textwidth}
		\centering
		\includegraphics*[width=\textwidth]{exp-snap-2.png}
	\end{subfigure}
    \begin{subfigure}[b]{0.24\textwidth}
		\centering
		\includegraphics*[width=\textwidth]{exp-snap-3.png}
	\end{subfigure}
    \begin{subfigure}[b]{0.24\textwidth}
		\centering
		\includegraphics*[width=\textwidth]{exp-snap-4.png}
	\end{subfigure}
	\hfill
    \vspace{0.5cm}
	\begin{subfigure}[b]{0.24\textwidth}
		\centering
		\includegraphics*[width=\textwidth]{snap-1.png}
	\end{subfigure}
	\begin{subfigure}[b]{0.24\textwidth}
		\centering
		\includegraphics*[width=\textwidth]{snap-2.png}
	\end{subfigure}
    \begin{subfigure}[b]{0.24\textwidth}
		\centering
		\includegraphics*[width=\textwidth]{snap-3.png}
	\end{subfigure}
    \begin{subfigure}[b]{0.24\textwidth}
		\centering
		\includegraphics*[width=\textwidth]{snap-4.png}
	\end{subfigure}
\caption{\red{\emph{Top row:} Snapshots of experiments, details to be added here. If possible, improve quality of images}}
\emph{Bottom row:} Oblique collision of two circular vortex rings at different 
(dimensionless) times \NOTE{GK: we should recal the value of $\tau$ for this run.}.
The superfluid vortex lines are represented by red tubes (the radius has been greatly 
exaggerated for visual purposes); the scaled normal fluid enstrophy 
$\bom^2/\bom^2_{max}$ is represented by the blue volume rendering. Here $\bom^2_{max}=50$ in dimensionless units \NOTE{GK: I guess that $\bom_{max}$ is changing in time. In that case just remove the last sentence.}. 
\label{fig:ring-coll-viz}
\end{figure*}

\paragraph*{Introduction.---} Reconnections are the fundamental events that
change the topology of the field lines in fluids and plasmas during their
time evolution. Reconnections
thus determine important physical properties, 
such as mixing and inter-scale energy transfer in fluids \cite{YaoHussainAnnRev2022}, 
or solar flares and tokamak instabilities
in plasmas \cite{Chapman2010}. The nature of reconnections is more clearly
studied if the field lines are concentrated in well-separated
filamentary structures:
vortices in fluids and magnetic flux tubes in plasmas. In superfluid 
helium this concentration is extreme, providing an ideal context:
superfluid vorticity is confined to vortex lines of atomic thickness 
(approximately $a_0 \approx 10^{-10}~\rm m$); a further simplification 
is that, unlike what happens in ordinary fluids, the
circulation of a superfluid vortex  is constrained to the quantized value
$\kappa=h/m=9.97 \times 10^{-8}~\rm m^2/s$, 
where $m$ is the mass of one helium atom and $h$ is Planck's constant. 

It was in this superfluid context that it was theoretically and experimentally recognized
\cite{nazarenko2003,bewley2008,paoletti2010,zuccherQuantumVortexReconnections2012a,villoisUniversalNonuniversalAspects2017a,galantucciCrossoverInteractionDriven2019a}
that reconnections share a universal property irrespective of the initial
condition: the minimum distance between reconnecting 
vortices, $\delta^{\pm}$, scales with time, $t$, according to the form
\begin{equation}
\label{eq:scaling}
	\delta^{\pm}(t) = A^{\pm} (\kappa|t-t_0|)^{1/2},
\end{equation} 
\noindent
where $t_0$ is the reconnection time, and the dimensionless
prefactors $A^-$ and $A^+$ refer respectively to before
($t<t_0$) and after ($t>t_0$) the reconnection. The same scaling law
was then found for reconnections in ordinary viscous fluids 
\cite{yaoSeparationScalingViscous2020}. In the case of a pure
superfluid at temperature $T=0~\rm K$, theoretical work based on
the Gross-Pitaevskii equation (GPE) has shown that
$A^+>A^-$, that is, after the reconnection, vortex lines move away from 
each others faster than in the initial approach; this result has been
related to irreversibility \cite{villoisIrreversibleDynamicsVortex2020,promentMatchingTheoryCharacterize2020}. Indeed, a geometrical constraint imposes that a piece of vortex length needs to be ``deleted'' during the reconnection process. In the GP model, this loss is possible by the emission of a rarefaction pulse created immediately after the reconnection
\cite{leadbeaterSoundEmissionDue2001b,zuccherQuantumVortexReconnections2012a} which removes some of the kinetic energy and momentum of the vortex configuration.
This vortex energy loss depends on
the ratio $A^+/A^-$, which in turns defines the approaching angle of collision
between the vortices, together with other several geometrical quantities \cite{villoisUniversalNonuniversalAspects2017a,promentMatchingTheoryCharacterize2020}. 

The temporal asymmetry $A^+>A^-$ can be thus interpreted as a non-trivial manifestation of irreversibility, as it originates from an ideal hydrodynamic process independent of the small-scale regularisation mechanism of the fluid.
%
Indeed, in classical fluid vortex reconnections, although the definition of $A^+$ is more delicate as circulation is not necessarily conserved, the same asymmetry $A^+>A^-$ was reported \cite{yaoSeparationScalingViscous2020}. Instead of the generation of rarefaction pulses, like in the case of $T=0$ superfluids, close to the reconnection, the classical fluid creates a series of thin secondary structures that can be then efficiently dissipated by viscous dissipation.

The case of superfluid helium is more intriguing. Most experiments are performed at temperatures $T>1~\rm K$, a regime in which in addition to quantum vortices, thermal excitations constitute a viscous liquid called the {\it normal fluid}. The normal fluid can steal energy from filaments and dissipate it by viscous effects, opening in that way more routes towards irreversibility. Modern visualisation techniques rely on active tracer particles to decorate superfluid vortices \cite{paoletti2008velocity,bewley2008,guo2014visualization,perettiDirectVisualizationQuantum2023}. Numerous studies have provided insight into the post-reconnection dynamics and the prefactor $A^{+}$, but much less is known about $A^-$ from experiments due to statistical likelihood of observation in the plane of view.

The aim of this Letter is to investigate the role played by the normal fluid in the reconnection dynamics. In particular, given the temperature dependence of the normal fluid's properties, we study experimentally and numerically the temperature dependence of the prefactors $A^+$ and $A^-$ and numerically investigate the energy injected in the normal fluid. 
%
To achieve this aim we need a more powerful model than the GPE to account not only for the dynamics of the
superfluid vortices, but also for the dynamics of the normal fluid. 
%The presence of the second, viscous normal fluid leads to a two-fluid 
%vortex reconnection.
We show that at non-zero temperatures Eq.~(\ref{eq:scaling}) and the
relation $A^+>A^-$ hold true, in agreement with experiments, revealing, for the 
first time, a temperature dependence of $A^+/A^-$. In addition, we
show that a vortex
reconnection represents an unusual kind of
punctuated energy injection into the normal fluid which acts alongside
the well-known (continual) friction.
When applied to superfluid turbulence, this last result implies that,
if the vortex line density (hence the frequency of reconnections) 
is large enough, vortex
reconnections can maintain the normal fluid in a perturbed state.
%%%%%%%%%%%%%%%%%%%%%%%%%%%%%%%%%%%%%%%%%%%%%%%%%%%%%%%%%%%

\paragraph*{Experimental Method.---} \NOTE{Add details of the experimental method here.} Note that GP simulations reported that particles trapped inside vortices do not drastically affect vortex reconnections \cite{GiuriatoQuantumVortexReconnections2020}. Whereas in the particle-vortex system, vortex energy and momentum are also transferred to particles, these additional mechanisms do not modify the approaching rates. This argument was also supported by a scaling symmetry of the system which allows to draw conclusion for length-scales relevant to experiments. 

%%%%%%%%%%%%%%%%%%%%%%%%%%%%%%%%%%%%%%%%%%%%%%%%%%%%%%%%%%%
\paragraph*{Numerical Method.---}We follow the approach of Schwarz \cite{schwarz1988} which exploits
the vast separation of length scales between the vortex core $a_0$ and any
other relevant distance, in particular the average distance between vortices,
$\ell$, in the case of turbulence. Vortex lines are described
as space curves $\s(\xi,t)$ where $\xi$ is arclength. The equation of motion 
of the vortex lines is
\begin{equation}
\label{eq:s}
	\dot{\s}(\xi,t) = \v_s + \frac{\beta}{(1+\beta)}\left[\v_{ns}\cdot \s'\right]\s' + \beta\s'\times\v_{ns}+\beta'\s'\times\left[\s'\times \v_{ns}\right],
\end{equation}
%
where $\dot{\s}=\partial\s/\partial t$, $\s'=\partial\s/\partial \xi$ 
is the unit tangent vector, 
$\v_n$ and $\v_s$ are the normal fluid and superfluid velocities at $\s$,
$\v_{ns}=\v_n - \v_s$, and $\beta$, $\beta'$ are temperature and Reynolds number dependent 
mutual friction coefficients \cite{galantucciNewSelfconsistentApproach2020b}. The normal fluid velocity $\v_n$ is described as a classical fluid obeying the incompressible ($\nabla\cdot\v_n=0$) Navier-Stokes equations:
%
\begin{equation}
\label{eq:vn}
	\frac{\partial \v_n}{\partial t} + (\v_n\cdot\nabla)\v_n = 
        -\frac{1}{\rho} \nabla p + \nu_n\nabla^2\v_n + \frac{\mathbf{F}_{ns}}{\rho_n},
\end{equation}
where $\mathbf{F}_{ns}$ is the mutual friction force that couples the normal fluid and the superfluid vortices, and acts as an internal injection mechanism. In Eq.~\eqref{eq:vn}, $\rho=\rho_n + \rho_s$, where $\rho_n$ and $\rho_s$ are the normal fluid and superfluid densities, $p$ is the pressure, 
and $\nu_n$ is the kinematic viscosity of the normal fluid. 
Equations~(\ref{eq:s}) and (\ref{eq:vn})
are solved in dimensionless form by rescaling them by the characteristic time $\tau$ and length $\lambda$. 
The algorithm for vortex reconnections is standard
\cite{baggaleySensitivityVortexFilament2012a}. We consider two distinct initial vortex configurations
at three temperatures $T=0~\rm K$, $1.9~\rm K$ and $2.1~\rm K$
corresponding to the superfluid fractions $\rho_s/\rho=100 \%$,
$58 \%$ and $26 \%$. To compare with experiments, the unit of length is set to $\lambda=1.59 \times 10^{-4}~\rm m$, and the time units to $\tau=0.183~\rm s$ at $T=0~\rm K$ and $1.9~\rm K$, and $\tau=0.242~\rm s$ at $T=2.1~\rm K$, see also \cite{SeeSupplementaryMaterials} for details.
%
All configurations lead to a vortex reconnection.
The first configuration consists of two vortex rings 
of (dimensionless) radius $R\approx 1$ in a tent-like shape
which collide obliquely 
making an initial angle $\alpha$ with the vertical direction, 
as shown in Fig.~\ref{fig:ring-coll-viz}, and,
schematically, in Fig.~\ref{fig:prefactors}.   
By changing the parameter $\alpha$, we create a sample of 12 realizations
at each temperature (again, see the Supplementary Material 
\cite{SeeSupplementaryMaterials} for details).
%We take 12 realizations of $\alpha$, such that 
%$\alpha\in\lbrace i\pi/13|i=1,\cdots,12\rbrace$. 
%The supplementary material 
%gives more details of our model and these configurations
%\cite{SeeSupplementaryMaterials}. 
The second configuration is the Hopf link, shown schematically in Fig.~\ref{fig:prefactors}. It consists of two perpendicular linked rings of radius $R\approx1$ with an offset in the $xy$-plane.
By changing the offset, we create a sample of 49 reconnections
at each temperature, as described in the Supplementary Material (SM)
\cite{SeeSupplementaryMaterials}.
%defined by parameters $\Delta l_x$ and $\Delta l_y$ 
%The offsets are chosen so that 
%$(\Delta l_x, \Delta l _y) \in \lbrace(0.125i,0.125j)|i,j=-3,\cdots,3 \rbrace$,
%providing us with a total of 49 reconnections for each temperature. 
In all cases, normal fluid structures generated by moving superfluid vortex rings \cite{kivotides-barenghi-samuels-2000}, are initially prepared to eliminate the transient phases (see SM for details).


\paragraph*{Scaling law. ---}In the experiment, two reconnections were observed where both $A^+$ and $A^-$ could be identified and calculated, at $T=1.65K$ and $T=2K$, plotted as orange triangles in Fig.~\ref{fig:prefactors}. Their corresponding minimal distances are displayed in the inset.
%
The pre-reconnection factor $A^-$ lies within the 0.4-0.6 range, consistent with the results of the numerics, and a clear temperature effect between a superfluid component majority and normal fluid component majority.
In the case of the Hopf link 
%the interaction between the two rings leads to a reconnection.  
we have performed 147 simulations 
(49 across 3 temperatures) as shown in Fig.~\ref{fig:minimum-distance} and
verified Eq.~(\ref{eq:scaling}) for the minimum distance $\delta^{\pm}$. 
The prefactors $A^{\pm}$ have been computed in the shaded region 
of the figure. In the pre-reconnection regime ($t<t_0$) we observe
a clear segregation of the values of $A^-$ due to temperature: 
the minimum distance grows more rapidly with time if the temperature is
lowered. 
In stark contrast, there is almost no memory of the temperature in the
post-reconnection regime ($t>t_0$). 
%\red{The lack of a temperature dependence for $A^+$ suggests
%that the normal fluid plays only a minor role in the
%dynamics of the post-reconnection regime.


At $T=0~{\rm K}$, our calculations for superfluid helium 
(black symbols in Fig.~\ref{fig:prefactors}) are in good agreement 
with previous results obtained with the GPE \cite{villoisIrreversibleDynamicsVortex2020}
(green diamonds), showing irreversible dynamics. In addition, the computed values of $A^-\approx 0.4$-$0.6$ at $T=0~{\rm K}$
are consistent with analytical calculations \cite{boue-etal-2013,rica-2019}. 
At non-zero temperatures, our results confirm the irreversibility of vortex reconnections 
observed at $T=0$ as $A^+$ is always larger than $A^-$. Importantly, this asymmetry 
is recovered in all our simulations, regardless of their initial condition.
%Our results confirm the irreversibility of vortex reconnections 
%\red{($A^+>A^-$)} at non-zero temperatures. 
%As shown in Fig.~\ref{fig:prefactors}, \red{our} $T=0~{\rm K}$ calculation 
%for superfluid helium is in good agreement with the GPE model for
%a condensate at $T=0$, which yields $A^-\approx 0.4$-$0.6$ 
%for both initial conditions \red{[ADD REF]}. 
The same asymmetry between
$A^+$ and $A^-$ at non-zero temperatures has been observed for reconnections in finite-temperature
Bose-Einstein condensatates \cite{allen2014}, although in this work the system
is not homogeneous (the condensate is confined by a harmonic trap) and the
thermal component is a ballistc gas, not a viscous fluid.
Note that the vortex reconnections in classical viscous fluids reported in %the classical Navier-Stokes equation
\cite{yaoSeparationScalingViscous2020} also display
a clear $1/2$ power-law scaling for the minimum distance
with $A^- \approx 0.3$-$0.4$, which again shows good agreement with our
results. The scaling law (Eq.~\ref{eq:scaling}) and the value of $A^-$ hence appear to have a universal character in vortex reconnections, independently of the nature
of the fluid, classical or quantum, and temperature.


%\paragraph*{\red{Energy injection. ---}}

\begin{figure}[t]
	\centering
	\begin{subfigure}[b]{0.45\textwidth}
		\centering
		\includegraphics*[width=\textwidth]{min-delta-with-exp.pdf}
		\caption{}
		\label{fig:minimum-distance}
	\end{subfigure}
	\begin{subfigure}[b]{0.45\textwidth}
		\centering
		\includegraphics*[width=\textwidth]{prefactors-with-exp.pdf}
		\caption{}
		\label{fig:prefactors}
	\end{subfigure}
\caption{\emph{(a)}: Time evolution of the (dimensionless)
minimum distance squared $\delta^2$ plotted versus (dimensionless) $\kappa (t-t_0)$ for
the Hopf link reconnections at $T=0K,1.9K$ and $2.1K$ (black, blue and red
respectively). The grey shaded areas
are the regions used to estimate the prefactors $A^{\pm}$. 
\emph{Inset:} Experimental superfluid helium data.
\emph{(b)}: Comparison of all prefactors: Hopf links (\emph{HL}, circles), 
ring collisions (\emph{RC}, stars with yellow outline), GPE-data 
from Villois \etal \cite{villoisIrreversibleDynamicsVortex2020} (green diamonds) and experimental results from this study (orange triangles). 
The shaded areas associated with each colour represent the convex hull of errors 
for each temperature. Schematic rendering of initial conditions are included.}
\end{figure}

\paragraph*{Energy injection. ---}
The normal fluid impacts the dynamics of reconnecting superfluid vortices via the temperature dependent mutual friction coefficients. Conversely, the motion of superfluid vortices involved in the reconnection process influence, significantly, the dynamics of the normal fluid. Figure~\ref{fig:energy-evol} indeed shows that the normal fluid energy, $E_n$,
suddenly increases at the reconnection time by an amount ($\approx 5\%$)
which is smaller but comparable to the continuous energy increase as vortex
lines approach each other. Indeed the
curvature $\zeta=|\s''|$ of the vortex line 
spikes at $t=t_0$ when
the reconnection cusp is created, and, in the first approximation \cite{galantucci-krstulovic-etal-2023},
the magnitude of the energy injected in the normal fluid per unit time $I$
is proportional to the strength of the mutual friction force $\mathbf{F}_{ns}$ which scales as
$|\mathbf{F}_{ns}(\s)|\propto|\dot{\s}-\v_n|\propto|\dot{\s}|\propto\zeta$. This sudden
transfer of energy
\cite{stasiakCrossComponentEnergyTransfer2024} from the superfluid
vortex configuration to the normal fluid is the origin of the 
small scale normal fluid enstrophy structures
which are visible in Fig.~\ref{fig:ring-coll-viz}.


%The prefactor ratio $A^+/A^-$ has been shown in recent literature 
%\cite{villoisIrreversibleDynamicsVortex2020,villoisUniversalNonuniversalAspects2017a,promentMatchingTheoryCharacterize2020} 
%to be of importance in describing fundamental properties of reconnections 
%using the Gross-Pitaevskii (GP) equation. Due to the \emph{ad-hoc} 
%nature of vortex reconnections in our helium model, it is not trivial 
%to derive a linear theory in the limit $\delta^{\pm}\rightarrow0$. 


\begin{figure}
	\centering
	\includegraphics*[width=0.45\textwidth]{energy-evolution.pdf}
	\caption{Normal fluid kinetic energy $E_n$ scaled by $E_n^0$ (the 
kinetic energy at $t=t_0$), plotted versus (dimensionless)
$\kappa (t-t_0)$ for the Hopf link reconnections. Black diamonds represent the simulations 
with minimum and maximum prefactor ratios $A^+/A^-$ at $T=1.9~\rm K$ and 
$T=2.1~\rm K$ respectively. \NOTE{GK: divide by $\lambda^2$ the legend of the x-axis.}}
	\label{fig:energy-evol}
\end{figure}

The total energy injected into the normal fluid by the reconnection,
$\Delta E_n$, which hereafter
we refer to as the energy jump, is defined as
\begin{equation}
	\Delta E_n = \max{\left[E_n(t>t_0)\right]} - E_n^0,
\end{equation} 
%
where $E_n^0=E_n(t_0)$ is the normal fluid kinetic energy at $t=t_0$.
Normalized energy jumps are plotted in
Fig.~\ref{fig:energy-jumps} as a function of
the ratio $A^+/A^-$. Here, we observe that the larger $A^+/A^-$ is,
the smaller the normal fluid excitation is.

The emission of the sound pulse at the vortex reconnection 
\cite{leadbeaterSoundEmissionDue2001b} which is typical of the GPE model
is absent in our incompressible hydrodynamic approach. To model
this effect, the change of vortex length, $\Delta L$, created by the
vortex reconnection algorithm is always negative by construction
\cite{baggaleySensitivityVortexFilament2012a},
because, in the local induction approximation to the Biot-Savart law, 
the superfluid incompressible kinetic energy, $E_s$, is proportional to the vortex length,
$L$. Such procedure ensures that at $T=0~{\rm K}$ when a reconnection occurs
$\Delta E_s\propto \Delta L < 0$. %Therefore, at $T=0~{\rm K}$, in the absence of any dissipative 
%normal fluid, $\Delta E_s\propto \Delta L$. 
Consequentially, in the absence of any dissipative normal fluid,
the superfluid energy 
$E_s$ that would be transferred to the sound pulse, normalized with its value $E_s^0$ at 
reconnection, is $-\Delta L/L_0$., 
%should exhibit the same characteristic behavior as the normal fluid energy 
%which is injected, $\Delta E_n/E_n^0$. 
If these normalized energy jumps (black diamonds
in Fig.~\ref{fig:energy-jumps}) are compared to the results obtained with the 
compressible GPE \cite{villoisIrreversibleDynamicsVortex2020} (purple squares)
we find a good agreement, confirming that the model we employ, 
%despite the ad-hoc algorithm employed to take into account the reconnection events
is suitable for the investigation of the feature of single reconnection events.
%Indeed, the black diamonds
%in Fig.~\ref{fig:energy-jumps} shows that our $T=0~{\rm K}$ simulations are
%in good agreement with the GPE results (purple squares).



\paragraph*{Implications for turbulence. ---}
Our numerical results have implications for our understanding of 
quantum turbulence \cite{BSS2023}.
A fully developed turbulent tangle of vortices is
characterized by its vortex line density $\mathcal{L}$ (vortex length
per unit volume); the frequency 
of vortex reconnections per unit volume is 
${f=(\kappa/6\pi)\mathcal{L}^{5/2}\ln(\mathcal{L}^{-1/2}/a_0)}$
\cite{barenghi2004}. From Fig.~\ref{fig:energy-evol} we estimate the 
normal fluid reconnection relaxation time $\tau_n$ as the time 
after reconnection at which the normal fluid energy $E_n/E_0$ has decayed
to the pre-reconnection level: in our dimensionless units, $\kappa \tau_n \approx 0.25$. 
Using this timescale, we estimate that
the average vortex line density that is required to sustain the normal fluid 
in a perturbed state via frequent vortex reconnections is approximately
$\mathcal{L} \approx 10^7$ to $10^8\mathrm{m}^{-2}$. Experiments in $^4$He
\cite{schwarz1981,milliken1982,roche2008,roche2007,Babuin2014} and in $^3$He
\cite{bradley2006} can achieve vortex line densities much larger than this.


\begin{figure}
	\centering
	\includegraphics*[width=0.48\textwidth]{energy-jump.pdf}
	\caption{Normalized energy jumps $\Delta E_n/E_n^0$ for Hopf link 
reconnections.The solid black diamonds are the normalized change in line length $\Delta L/L_0$ in the
$T=0~\rm K$ case. Blue and red circle correspond to $T=1.9K$ and $T=2.1K$,respectively. The purple squares are from GPE simulations of
Villois et al. \cite{villoisIrreversibleDynamicsVortex2020}.}
	\label{fig:energy-jumps}
\end{figure}

\paragraph*{Conclusions.---} We have conducted an experiment using active particle tracers and a statistical numerical study of vortex reconnections in a wide range of temperatures using a model of $^4$He which accounts for the coupled dynamics of superfluid and normal fluid components.
We have verified the scaling law of the minimum vortex distance 
$\delta^{\pm}=A^{\pm} (\kappa |t-t_0|)^{1/2}$ and found that the approach prefactor $A^-$ has a clear temperature dependence independent of the geometry in both experiments and numerics, in contrast to the
separation prefactor $A^+$. The prefactors are in good agreement
with GPE simulations \cite{villoisIrreversibleDynamicsVortex2020,allen2014} 
and classical fluid reconnections \cite{yaoSeparationScalingViscous2020}
revealing that vortex reconnections display a universal behaviour regardless of the nature
of the fluid (classical or quantum) and of temeperature. It is worth noting that the behaviour, as a function of $A^+/A^-$, of 
the energy injected in the normal fluid (at $T>0$) and of the energy transferred to sound (at $T=0$)
\cite{villoisIrreversibleDynamicsVortex2020,leadbeaterSoundEmissionDue2001b} is dissimilar: the former decreases as $A^+/A^-$
increases, the latter the opposite. This likely arises from the distinct physics governing the loss of superfluid 
incompressible kinetic energy: mutual friction at $T>0$, quantum pressure at $T=0$.
We have also found that a reconnection event suddenly injects an amount of energy 
into the normal fluid which is comparable to the energy transferred by friction
during the vortex approach. Applying these results to turbulence, we have
compared the decay time of the normal fluid structures created by a
reconnection to the frequency of reconnections in a vortex tangle, and argued
that, if the vortex line density is large enough, these punctuated
energy injections should sustain the normal fluid in a perturbed state, which may lead to a new type of turbulence.

\bibliography{references3}% Produces the bibliography via BibTeX.

% % ****** Start of file apssamp.tex ******
%
%   This file is part of the APS files in the REVTeX 4.2 distribution.
%   Version 4.2a of REVTeX, December 2014
%
%   Copyright (c) 2014 The American Physical Society.
%
%   See the REVTeX 4 README file for restrictions and more information.
%
% TeX'ing this file requires that you have AMS-LaTeX 2.0 installed
% as well as the rest of the prerequisites for REVTeX 4.2
%
% See the REVTeX 4 README file
% It also requires running BibTeX. The commands are as follows:
%
%  1)  latex apssamp.tex
%  2)  bibtex apssamp
%  3)  latex apssamp.tex
%  4)  latex apssamp.tex
%
\documentclass[%
 reprint,
% superscriptaddress,
%groupedaddress,
%unsortedaddress,
%runinaddress,
%frontmatterverbose, 
%preprint,
%preprintnumbers,
%nofootinbib,
%nobibnotes,
%bibnotes,
 amsmath,amssymb,
 aps,
 prl,
%pra,
% prb,
% rmp,
%prstab,
%prstper,
%floatfix,
]{revtex4-2}

\usepackage{graphicx}% Include figure files
\usepackage{dcolumn}% Align table columns on decimal point
\usepackage{bm}% bold math
\usepackage{blindtext}
\usepackage{float}
\usepackage{caption}
\usepackage{cleveref}
\usepackage{subcaption}
\usepackage{xcolor}
%\usepackage{hyperref}% add hypertext capabilities
%\usepackage[mathlines]{lineno}% Enable numbering of text and display math
%\linenumbers\relax % Commence numbering lines

%\usepackage[showframe,%Uncomment any one of the following lines to test 
%%scale=0.7, marginratio={1:1, 2:3}, ignoreall,% default settings
%%text={7in,10in},centering,
%%margin=1.5in,
%%total={6.5in,8.75in}, top=1.2in, left=0.9in, includefoot,
%%height=10in,a5paper,hmargin={3cm,0.8in},
%]{geometry}


\newcommand{\etal}{{\it et al.}~}
\newcommand{\bom}{\boldsymbol{\omega}}

% \newcommand{\sd}[3][\null]{\ensuremath{\dfrac{\d^{#1} #2}{\d #3^{#1}}}}%Standard derivative 
% \newcommand{\pd}[3][\null]{\ensuremath{\dfrac{\partial^{#1} #2}{\partial #3^{#1}}}}%Partial derivative
% \newcommand{\matd}[2][\null]{\ensuremath{\dfrac{\mathrm{D}^{#1} #2}{\mathrm{D} t^{#1}}}} %Material derivative

\def \s{\mathbf{s}}
\def \v{\mathbf{v}}
\def \x{\mathbf{x}}
\def \r{\mathbf{r}}
\def \k{\mathbf{k}}

\def \cm{\mathrm{cm}}
\def \cms{\mathrm{cm/s}}
\def \sec{\mathrm{s}}
\def \K{\mathrm{K}}

\def\red#1{\textcolor{red}{#1}}
\def\blue#1{\textcolor{blue}{#1}}
%%%%%%%%%%%%%%%%%%%%%%%%%%%%%%%%%



\begin{document}

\preprint{APS/123-QED}

\title{Superfluid vortex reconnections at non-zero temperatures}

\author{P. Z. Stasiak}
\author{A. Baggaley}
\author{C.F. Barenghi}
\affiliation{School of Mathematics, Statistics and Physics, Newcastle University, Newcastle upon Tyne, NE1 7RU, United Kingdom}

\author{G. Krstulovic}
\affiliation{Universit\'e C\^ote d'Azur, Observatoire de la C\^ote d'Azur, CNRS,Laboratoire Lagrangre, Boulevard de l'Observatoire CS 34229 - F 06304 NICE Cedex 4, France}

\author{L. Galantucci}
\affiliation{Istituto per le Applicazioni del Calcolo ``M. Picone" IAC CNR, Via dei Taurini 19, 00185 Roma, Italy}

\author{\red{W. Guo}}
\author{\red{Y. Xin}}
\author{\red{Y. Alihosseini}}
\affiliation{\red{Department of Mechanical Engineering, FAMU-FSU College of Engineering, Florida State University, Tallahassee, Florida 32310, USA}}
\date{\today}% It is always \today, today,
             %  but any date may be explicitly specified

\begin{abstract}
The minimum separation between reconnecting vortices
in fluids and superfluids obeys a universal scaling law with respect to time.
The pre-reconnection and the post-reconnection prefactors 
of this scaling law are different, a property related to irreversibility.
\red{Using experiments and a} numeric model which fully accounts for the independent dynamics of the superfluid
vortex lines and the thermal normal fluid component, we determine the temperature
dependence of these prefactors. We also \red{numerically} show that each vortex reconnection event
represents a sudden injection of energy in the normal fluid. Finally we argue that
in a turbulent flow, these punctuated energy injections can sustain the normal fluid
in a perturbed state, provided that the density of superfluid vortices is large enough.
\end{abstract}

%\keywords{Suggested keywords}%Use showkeys class option if keyword
   %display desired
\maketitle

\begin{figure*}[t]
	\centering
	\begin{subfigure}[b]{0.24\textwidth}
		\centering
		\includegraphics*[width=\textwidth]{exp-snap-1.png}
	\end{subfigure}
	\begin{subfigure}[b]{0.24\textwidth}
		\centering
		\includegraphics*[width=\textwidth]{exp-snap-2.png}
	\end{subfigure}
    \begin{subfigure}[b]{0.24\textwidth}
		\centering
		\includegraphics*[width=\textwidth]{exp-snap-3.png}
	\end{subfigure}
    \begin{subfigure}[b]{0.24\textwidth}
		\centering
		\includegraphics*[width=\textwidth]{exp-snap-4.png}
	\end{subfigure}
	\hfill
    \vspace{0.5cm}
	\begin{subfigure}[b]{0.24\textwidth}
		\centering
		\includegraphics*[width=\textwidth]{snap-1.png}
	\end{subfigure}
	\begin{subfigure}[b]{0.24\textwidth}
		\centering
		\includegraphics*[width=\textwidth]{snap-2.png}
	\end{subfigure}
    \begin{subfigure}[b]{0.24\textwidth}
		\centering
		\includegraphics*[width=\textwidth]{snap-3.png}
	\end{subfigure}
    \begin{subfigure}[b]{0.24\textwidth}
		\centering
		\includegraphics*[width=\textwidth]{snap-4.png}
	\end{subfigure}
\caption{\red{\emph{Top row:} Snapshots of experiments, details to be added here.}}
\emph{Bottom row:} Oblique collision of two circular vortex rings at different 
(dimensionless) times.
The superfluid vortex lines are represented by red tubes (the radius has been greatly 
exaggerated for visual purposes); the scaled normal fluid enstrophy 
$\bom^2/\bom^2_{max}$ is represented by the blue volume rendering. Here $\bom^2_{max}=50$ in dimensionless units. 
\label{fig:ring-coll-viz}
\end{figure*}

\paragraph*{Introduction.---} Reconnections are the fundamental events that
change the topology of the field lines in fluids and plasmas during their
time evolution. Reconnections
thus determine important physical properties, 
such as mixing and inter-scale energy transfer in fluids \cite{YaoHussainAnnRev2022}, 
or solar flares and tokamak instabilities
in plasmas \cite{Chapman2010}. The nature of reconnections is more clearly
studied if the field lines are concentrated in well-separated
filamentary structures:
vortices in fluids and magnetic flux tubes in plasmas. In superfluid 
helium this concentration is extreme, providing an ideal context:
superfluid vorticity is confined to vortex lines of atomic thickness 
(approximately $a_0 \approx 10^{-10}~\rm m$); a further simplification 
is that, unlike what happens in ordinary fluids, the
circulation of a superfluid vortex  is constrained to the quantized value
$\kappa=h/m=9.97 \times 10^{-8}~\rm m^2/s$, 
where $m$ is the mass of one helium atom and $h$ is Planck's constant. 

It was in this superfluid context that it was theoretically and experimentally recognized
\cite{nazarenko2003,bewley2008,paoletti2010,zuccherQuantumVortexReconnections2012a,villoisUniversalNonuniversalAspects2017a,galantucciCrossoverInteractionDriven2019a}
that reconnections share a universal property irrespective of the initial
condition: the minimum distance between reconnecting 
vortices, $\delta^{\pm}$, scales with time, $t$, according to the form
\begin{equation}
\label{eq:scaling}
	\delta^{\pm}(t) = A^{\pm} (\kappa|t-t_0|)^{1/2},
\end{equation} 
\noindent
where $t_0$ is the reconnection time, and the dimensionless
prefactors $A^-$ and $A^+$ refer respectively to before
($t<t_0$) and after ($t>t_0$) the reconnection. The same scaling law
was then found for reconnections in ordinary viscous fluids 
\cite{yaoSeparationScalingViscous2020}. In the case of a pure
superfluid at temperature $T=0~\rm K$, theoretical work based on
the Gross-Pitaevskii equation (GPE) has shown that
$A^+>A^-$, that is, after the reconnection, vortex lines move away from 
each others faster than in the initial approach; this result has been
related to irreversibility \cite{villoisIrreversibleDynamicsVortex2020},
and is caused by a rarefaction pulse created immediately after the reconnection
\cite{leadbeaterSoundEmissionDue2001b,zuccherQuantumVortexReconnections2012a} which
removes some of the kinetic energy of the vortex configuration.
This acoustic energy loss depends on
the ratio $A^+/A^-$, which in turns depends on the angle of collision
between the vortices \cite{villoisIrreversibleDynamicsVortex2020}. 

However most helium experiments are performed at temperatures $T>1~\rm K$, a regime in which thermal excitations form a fluid called the {\it normal fluid} which provides a viscous route to irreversibility.\red{Modern visualisation techniques rely on active tracer particles to decorate superfluid vortices \cite{paoletti2008velocity,bewley2008,guo2014visualization}. Numerous studies have provided insight into the post-reconnection dynamics and the prefactor $A^{+}$, but much less is known about $A^-$ from experiments due to statistical likelihood of observation in the plane of view.}

The aim of this Letter is to investigate the role played by the normal fluid in the reconnection dynamics. In particular, given the
temperature dependence of the normal fluid's properties, we study \red{experimentally and numerically} the temperature
dependence of the prefactors $A^+$ and $A^-$ and \red{numerically investigate} the energy injected in the normal fluid. 

To achieve this aim we need a more powerful model than the GPE to account not only for the dynamics of the
superfluid vortices, but also for the dynamics of the normal fluid. 
%The presence of the second, viscous normal fluid leads to a two-fluid 
%vortex reconnection.
We show that at non-zero temperatures Eq.~(\ref{eq:scaling}) and the
relation $A^+>A^-$ hold true, in agreement with experiments, revealing, for the 
first time, a temperature dependence of $A^+/A^-$. In addition, we
show that a vortex
reconnection represents an unusual kind of
punctuated energy injection into the normal fluid which acts alongside
the well-known (continual) friction.
When applied to superfluid turbulence, this last result implies that,
if the vortex line density (hence the frequency of reconnections) 
is large enough, vortex
reconnections can maintain the normal fluid in a perturbed state.
%%%%%%%%%%%%%%%%%%%%%%%%%%%%%%%%%%%%%%%%%%%%%%%%%%%%%%%%%%%
\red{
\paragraph*{Experimental Method.---} Add details of the experimental method here.
}
%%%%%%%%%%%%%%%%%%%%%%%%%%%%%%%%%%%%%%%%%%%%%%%%%%%%%%%%%%%
\paragraph*{Numerical Method.---}We follow the approach of Schwarz \cite{schwarz1988} which exploits
the vast separation of length scales between the vortex core $a_0$ and any
other relevant distance, in particular the average distance between vortices,
$\ell$, in the case of turbulence. Vortex lines are described
as space curves $\s(\xi,t)$ where $\xi$ is arclength. The equation of motion 
of the vortex lines is
\begin{equation}
\label{eq:s}
	\dot{\s}(\xi,t) = \v_s + \frac{\beta}{(1+\beta)}\left[\v_{ns}\cdot \s'\right]\s' + \beta\s'\times\v_{ns}+\beta'\s'\times\left[\s'\times \v_{ns}\right],
\end{equation}

\noindent
where $\dot{\s}=\partial\s/\partial t$, $\s'=\partial\s/\partial \xi$ 
is the unit tangent vector, 
$\v_n$ and $\v_s$ are the normal fluid and superfluid velocities at $\s$,
$\v_{ns}=\v_n - \v_s$, 
and $\beta$, $\beta'$ are temperature and Reynolds number dependent 
mutual friction coefficients \cite{galantucciNewSelfconsistentApproach2020b}. Superfluid vortices are coupled to a classical description of the 
incompressible ($\nabla\cdot\v_n=0$) normal fluid via the mutual 
friction force $\mathbf{F}_{ns}$, \red{an internal injection in the Navier-Stokes model}

\begin{equation}
\label{eq:vn}
	\frac{\partial \v_n}{\partial t} + (\v_n\cdot\nabla)\v_n = 
        -\frac{1}{\rho} \nabla p + \nu_n\nabla^2\v_n + \frac{\mathbf{F}_{ns}}{\rho_n},
\end{equation}

\noindent
where $\rho=\rho_n + \rho_s$ is the total density, $\rho_n$ and 
$\rho_s$ are the normal fluid and superfluid densities, $p$ is the pressure, 
and $\nu_n$ is the kinematic viscosity of the normal fluid. 
Equations~(\ref{eq:s}) and (\ref{eq:vn})
are solved in dimensionless form using the length unit 
$\lambda=1.59 \times 10^{-4}~\rm m$ and the time unit $\tau$ defined next, 
see also \cite{SeeSupplementaryMaterials} 
for details.
The algorithm for vortex reconnections is standard
\cite{baggaleySensitivityVortexFilament2012a}. We consider two distinct initial vortex configurations
at three temperatures $T=0~\rm K$, $1.9~\rm K$ and $2.1~\rm K$
corresponding to the superfluid fractions $\rho_s/\rho=100 \%$,
$58 \%$ and $26 \%$. To make the equations dimensionless,
we use the time units $\tau=0.183~\rm s$ at $T=0~\rm K$
and $1.9~\rm K$, and $\tau=0.242~\rm s$ at $T=2.1~\rm K$. 
All configurations lead to a vortex reconnection.
The first configuration consists of two vortex rings 
of (dimensionless) radius $R\approx 1$ in a tent-like shape
which collide obliquely 
making an initial angle $\alpha$ with the vertical direction, 
as shown in Fig.~\ref{fig:ring-coll-viz}, and,
schematically, in Fig.~\ref{fig:prefactors}.   
By changing the parameter $\alpha$, we create a sample of 12 realizations
at each temperature (again, see the Supplementary Material 
\cite{SeeSupplementaryMaterials} for details).
%We take 12 realizations of $\alpha$, such that 
%$\alpha\in\lbrace i\pi/13|i=1,\cdots,12\rbrace$. 
%The supplementary material 
%gives more details of our model and these configurations
%\cite{SeeSupplementaryMaterials}. 
The second configuration is the Hopf link, shown schematically in Fig.~\ref{fig:prefactors}. It consists of two perpendicular linked rings of radius $R\approx1$ with an offset in the $xy$-plane.
By changing the offset, we create a sample of 49 reconnections
at each temperature, as described in the Supplementary Material
\cite{SeeSupplementaryMaterials}.
%defined by parameters $\Delta l_x$ and $\Delta l_y$ 
%The offsets are chosen so that 
%$(\Delta l_x, \Delta l _y) \in \lbrace(0.125i,0.125j)|i,j=-3,\cdots,3 \rbrace$,
%providing us with a total of 49 reconnections for each temperature. 
In all cases, normal fluid structures
in the form of rings \cite{kivotides-barenghi-samuels-2000} are 
initially superimposed to match the vortex lines, eliminating the transient 
phase of generating these normal fluid structures.


\paragraph*{Scaling law. ---}\red{In the experiment, two reconnections were observed where both $A^+$ and $A^-$ could be identified and calculated, at $T=1.65K$ and $T=2K$, plotted as orange triangles in Fig.~\ref{fig:prefactors}. The pre-reconnection factor $A^-$ lies within the 0.4-0.6 range, consistent with the results of the numerics, and a clear temperature effect between a superfluid component majority and normal fluid component majority.}
In the case of the Hopf link 
%the interaction between the two rings leads to a reconnection.  
we have performed 147 simulations 
(49 across 3 temperatures) as shown in Fig.~\ref{fig:minimum-distance} and
verified Eq.~(\ref{eq:scaling}) for the minimum distance $\delta^{\pm}$. 
The prefactors $A^{\pm}$ have been computed in the shaded region 
of the figure. In the pre-reconnection regime ($t<t_0$) we observe
a clear segregation of the values of $A^-$ due to temperature: 
the minimum distance grows more rapidly with time if the temperature is
lowered. 
In stark contrast, there is no memory of the temperature in the
post-reconnection regime ($t>t_0$). 
%\red{The lack of a temperature dependence for $A^+$ suggests
%that the normal fluid plays only a minor role in the
%dynamics of the post-reconnection regime.


At $T=0~{\rm K}$ our calculations for superfluid helium 
(black symbols in Fig.~\ref{fig:prefactors}) are in good agreement 
with previous results obtained with the GPE \cite{villoisIrreversibleDynamicsVortex2020}
(green diamonds), showing irreversible dynamics. In addition, the computed values of $A^-\approx 0.4$-$0.6$ at $T=0~{\rm K}$
are consistent with analytical calculations \cite{boue-etal-2013,rica-2019}. 
At non-zero temperatures, our results confirm the irreversibility of vortex reconnections 
observed at $T=0$ as $A^+$ is always larger than $A^-$. Importantly, this asymmetry 
is recovered in all our simulations, regardless of their initial condition.
%Our results confirm the irreversibility of vortex reconnections 
%\red{($A^+>A^-$)} at non-zero temperatures. 
%As shown in Fig.~\ref{fig:prefactors}, \red{our} $T=0~{\rm K}$ calculation 
%for superfluid helium is in good agreement with the GPE model for
%a condensate at $T=0$, which yields $A^-\approx 0.4$-$0.6$ 
%for both initial conditions \red{[ADD REF]}. 
The same asymmetry between
$A^+$ and $A^-$ at non-zero temperatures has been observed for reconnections in finite-temperature
Bose-Einstein condensatates \cite{allen2014}, although in this work the system
is not homogeneous (the condensate is confined by a harmonic trap) and the
thermal component is a ballistc gas, not a viscous fluid.
Recent investigations of vortex reconnections in classical viscous fluids %the classical Navier-Stokes equation
\cite{yaoSeparationScalingViscous2020} also display
a clear $1/2$ power-law scaling for the minimum distance
with $A^- \approx 0.3$-$0.4$, which again shows good agreement with our
results. The scaling law (Eq.~\ref{eq:scaling}) and the value of $A^-$ hence appear 
to have a universal character in vortex reconnections, independently of the nature
of the fluid, classical or quantum, and temperature.


%\paragraph*{\red{Energy injection. ---}}

\begin{figure}[t]
	\centering
	\begin{subfigure}[b]{0.45\textwidth}
		\centering
		\includegraphics*[width=\textwidth]{min-delta.pdf}
		\caption{}
		\label{fig:minimum-distance}
	\end{subfigure}
	\begin{subfigure}[b]{0.45\textwidth}
		\centering
		\includegraphics*[width=\textwidth]{prefactors-with-exp.pdf}
		\caption{}
		\label{fig:prefactors}
	\end{subfigure}
\caption{\emph{(a)}: Time evolution of the (dimensionless)
minimum distance squared $\delta^2$ plotted versus (dimensionless) $\kappa (t-t_0)$ for
the Hopf link reconnections at $T=0K,1.9K$ and $2.1K$ (black, blue and red
respectively). The grey shaded areas
are the regions used to estimate the prefactors $A^{\pm}$. 
\emph{Inset:} Values of the separation prefactor $A^+$ and approach prefactors $A^-$. 
The dashed line corresponds to $A^+ = A^-$
\emph{(b)}: Comparison of all prefactors: Hopf links (\emph{HL}, circles), 
ring collisions (\emph{RC}, stars with yellow outline), GPE-data 
from Villois \etal \cite{villoisIrreversibleDynamicsVortex2020} (green diamonds) and experimental results from this study (orange triangles). 
The shaded areas associated with each colour represent the convex hull of errors 
for each temperature. Schematic rendering of initial conditions are included.}
\end{figure}

\paragraph*{Energy injection. ---}
The normal fluid impacts the dynamics of reconnecting superfluid vortices via the temperature dependent mutual friction coefficients. Conversely, the motion of superfluid vortices involved in the reconnection process influence, significantly, the dynamics of the normal fluid. Fig.~\ref{fig:energy-evol} indeed shows that the normal fluid energy, $E_n$,
suddenly increases at the reconnection time by an amount ($\approx 5\%$)
which is smaller but comparable to the continuous energy increase as vortex
lines approach each other. Indeed the
curvature $\zeta=|\s''|$ of the vortex line 
spikes at $t=t_0$ when
the reconnection cusp is created, and, in the first approximation \cite{galantucci-krstulovic-etal-2023},
the magnitude of the energy injected in the normal fluid per unit time $I$
is proportional to the strength of the mutual friction force $\mathbf{F}_{ns}$ which scales as
$|\mathbf{F}_{ns}(\s)|\propto|\dot{\s}-\v_n|\propto|\dot{\s}|\propto\zeta$. This sudden
transfer of energy
\cite{stasiakCrossComponentEnergyTransfer2024} from the superfluid
vortex configuration to the normal fluid is the origin of the 
small scale normal fluid enstrophy structures
which are visible in Fig.~\ref{fig:ring-coll-viz}.


%The prefactor ratio $A^+/A^-$ has been shown in recent literature 
%\cite{villoisIrreversibleDynamicsVortex2020,villoisUniversalNonuniversalAspects2017a,promentMatchingTheoryCharacterize2020} 
%to be of importance in describing fundamental properties of reconnections 
%using the Gross-Pitaevskii (GP) equation. Due to the \emph{ad-hoc} 
%nature of vortex reconnections in our helium model, it is not trivial 
%to derive a linear theory in the limit $\delta^{\pm}\rightarrow0$. 


\begin{figure}
	\centering
	\includegraphics*[width=0.45\textwidth]{energy-evolution.pdf}
	\caption{Normal fluid kinetic energy $E_n$ scaled by $E_n^0$ (the 
kinetic energy at $t=t_0$), plotted versus (dimensionless)
$\kappa (t-t_0)$ for the Hopf link reconnections. Black diamonds represent the simulations 
with minimum and maximum prefactor ratios $A^+/A^-$ at $T=1.9~\rm K$ and 
$T=2.1~\rm K$ respectively.}
	\label{fig:energy-evol}
\end{figure}

The total energy injected into the normal fluid by the reconnection,
$\Delta E_n$, which hereafter
we refer to as the energy jump, is defined as

\begin{equation}
	\Delta E_n = \max{\left[E_n(t>t_0)\right]} - E_n^0,
\end{equation} 

\noindent
where $E_n^0=E_n(t_0)$ is the normal fluid kinetic energy at $t=t_0$.
Normalized energy jumps are plotted in
Fig.~\ref{fig:energy-jumps} as a function of
the ratio $A^+/A^-$. Here, we observe that the larger $A^+/A^-$ is 
(i.e the more the reconnection is anti-parallel \cite{villoisIrreversibleDynamicsVortex2020}),
the smaller the normal fluid excitation is.

The emission of the sound pulse at the vortex reconnection 
\cite{leadbeaterSoundEmissionDue2001b} which is typical of the GPE model
is absent in our incompressible hydrodynamic approach. To model
this effect, the change of vortex length, $\Delta L$, created by the
vortex reconnection algorithm is always negative by construction
\cite{baggaleySensitivityVortexFilament2012a},
because, in the local induction approximation to the Biot-Savart law, 
the superfluid incompressible kinetic energy, $E_s$, is proportional to the vortex length,
$L$. Such procedure ensures that at $T=0~{\rm K}$ when a reconnection occurs
$\Delta E_s\propto \Delta L < 0$. %Therefore, at $T=0~{\rm K}$, in the absence of any dissipative 
%normal fluid, $\Delta E_s\propto \Delta L$. 
Consequentially, in the absence of any dissipative normal fluid,
the superfluid energy 
$E_s$ that would be transferred to the sound pulse, normalized with its value $E_s^0$ at 
reconnection, is $-\Delta L/L_0$., 
%should exhibit the same characteristic behavior as the normal fluid energy 
%which is injected, $\Delta E_n/E_n^0$. 
If these normalized energy jumps (black diamonds
in Fig.~\ref{fig:energy-jumps}) are compared to the results obtained with the 
compressible GPE \cite{villoisIrreversibleDynamicsVortex2020} (purple squares)
we find a good agreement, confirming that the model we employ, 
%despite the ad-hoc algorithm employed to take into account the reconnection events
is suitable for the investigation of the feature of single reconnection events.
%Indeed, the black diamonds
%in Fig.~\ref{fig:energy-jumps} shows that our $T=0~{\rm K}$ simulations are
%in good agreement with the GPE results (purple squares).



\paragraph*{Implications for turbulence. ---}
Our numerical results have implications for our understanding of 
quantum turbulence \cite{BSS2023}.
A fully developed turbulent tangle of vortices is
characterized by its vortex line density $\mathcal{L}$ (vortex length
per unit volume); the frequency 
of vortex reconnections per unit volume is 
${f=(\kappa/6\pi)\mathcal{L}^{5/2}\ln(\mathcal{L}^{-1/2}/a_0)}$
\cite{barenghi2004}. From Fig.~\ref{fig:energy-evol} we estimate the 
normal fluid reconnection relaxation time $\tau_n$ as the time 
after reconnection at which the normal fluid energy $E_n/E_0$ has decayed
to the pre-reconnection level: in our dimensionless units, $\kappa \tau_n \approx 0.25$. 
Using this timescale, we estimate that
the average vortex line density that is required to sustain the normal fluid 
in a perturbed state via frequent vortex reconnections is approximately
$\mathcal{L} \approx 10^7$ to $10^8\mathrm{m}^{-2}$. Experiments in $^4$He
\cite{schwarz1981,milliken1982,roche2008,roche2007,Babuin2014} and in $^3$He
\cite{bradley2006} can achieve vortex line densities much larger than this.


\begin{figure}
	\centering
	\includegraphics*[width=0.48\textwidth]{energy-jump.pdf}
	\caption{Normalized energy jumps $\Delta E_n/E_n^0$ for Hopf link 
reconnections.The solid black diamonds are the normalized change in line length $\Delta L/L_0$ in the
$T=0~\rm K$ case. Blue and red circle correspond to $T=1.9K$ and $T=2.1K$,respectively. The purple squares are from GPE simulations of
Villois et al. \cite{villoisIrreversibleDynamicsVortex2020}.}
	\label{fig:energy-jumps}
\end{figure}

\paragraph*{Conclusions.---} \red{We have conducted an experiment using active particle tracers} and a statistical numerical study
of vortex reconnections in a wide range of temperatures using a model of $^4$He which accounts for the coupled dynamics of superfluid and normal fluid components.
We have verified the scaling law of the minimum vortex distance 
$\delta^{\pm}=A^{\pm} (\kappa |t-t_0|)^{1/2}$ and found that the approach prefactor $A^-$ has a clear temperature dependence independent of the geometry \red{in both experiments and numerics}, in contrast to the
separation prefactor $A^+$. The prefactors are in good agreement
with GPE simulations \cite{villoisIrreversibleDynamicsVortex2020,allen2014} 
and classical fluid reconnections \cite{yaoSeparationScalingViscous2020}
revealing that vortex reconnections display a universal behaviour regardless of the nature
of the fluid (classical or quantum) and of temeperature. It is worth noting that the behaviour, as a function of $A^+/A^-$, of 
the energy injected in the normal fluid (at $T>0$) and of the energy transferred to sound (at $T=0$)
\cite{villoisIrreversibleDynamicsVortex2020,leadbeaterSoundEmissionDue2001b} is dissimilar: the former decreases as $A^+/A^-$
increases, the latter the opposite. This likely arises from the distinct physics governing the loss of superfluid 
incompressible kinetic energy: mutual friction at $T>0$, quantum pressure at $T=0$.
We have also found that a reconnection event suddenly injects an amount of energy 
into the normal fluid which is comparable to the energy transferred by friction
during the vortex approach. Applying these results to turbulence, we have
compared the decay time of the normal fluid structures created by a
reconnection to the frequency of reconnections in a vortex tangle, and argued
that, if the vortex line density is large enough, these punctuated
energy injections should sustain the normal fluid in a perturbed state.

\bibliography{references3}% Produces the bibliography via BibTeX.

% % ****** Start of file apssamp.tex ******
%
%   This file is part of the APS files in the REVTeX 4.2 distribution.
%   Version 4.2a of REVTeX, December 2014
%
%   Copyright (c) 2014 The American Physical Society.
%
%   See the REVTeX 4 README file for restrictions and more information.
%
% TeX'ing this file requires that you have AMS-LaTeX 2.0 installed
% as well as the rest of the prerequisites for REVTeX 4.2
%
% See the REVTeX 4 README file
% It also requires running BibTeX. The commands are as follows:
%
%  1)  latex apssamp.tex
%  2)  bibtex apssamp
%  3)  latex apssamp.tex
%  4)  latex apssamp.tex
%
\documentclass[%
 reprint,
% superscriptaddress,
%groupedaddress,
%unsortedaddress,
%runinaddress,
%frontmatterverbose, 
%preprint,
%preprintnumbers,
%nofootinbib,
%nobibnotes,
%bibnotes,
 amsmath,amssymb,
 aps,
 prl,
%pra,
% prb,
% rmp,
%prstab,
%prstper,
%floatfix,
]{revtex4-2}

\usepackage{graphicx}% Include figure files
\usepackage{dcolumn}% Align table columns on decimal point
\usepackage{bm}% bold math
\usepackage{blindtext}
\usepackage{float}
\usepackage{caption}
\usepackage{cleveref}
\usepackage{subcaption}
\usepackage{xcolor}
%\usepackage{hyperref}% add hypertext capabilities
%\usepackage[mathlines]{lineno}% Enable numbering of text and display math
%\linenumbers\relax % Commence numbering lines

%\usepackage[showframe,%Uncomment any one of the following lines to test 
%%scale=0.7, marginratio={1:1, 2:3}, ignoreall,% default settings
%%text={7in,10in},centering,
%%margin=1.5in,
%%total={6.5in,8.75in}, top=1.2in, left=0.9in, includefoot,
%%height=10in,a5paper,hmargin={3cm,0.8in},
%]{geometry}


\newcommand{\etal}{{\it et al.}~}
\newcommand{\bom}{\boldsymbol{\omega}}

% \newcommand{\sd}[3][\null]{\ensuremath{\dfrac{\d^{#1} #2}{\d #3^{#1}}}}%Standard derivative 
% \newcommand{\pd}[3][\null]{\ensuremath{\dfrac{\partial^{#1} #2}{\partial #3^{#1}}}}%Partial derivative
% \newcommand{\matd}[2][\null]{\ensuremath{\dfrac{\mathrm{D}^{#1} #2}{\mathrm{D} t^{#1}}}} %Material derivative

\def \s{\mathbf{s}}
\def \v{\mathbf{v}}
\def \x{\mathbf{x}}
\def \r{\mathbf{r}}
\def \k{\mathbf{k}}

\def \cm{\mathrm{cm}}
\def \cms{\mathrm{cm/s}}
\def \sec{\mathrm{s}}
\def \K{\mathrm{K}}

\def\red#1{\textcolor{red}{#1}}
\def\blue#1{\textcolor{blue}{#1}}
%%%%%%%%%%%%%%%%%%%%%%%%%%%%%%%%%



\begin{document}

\preprint{APS/123-QED}

\title{Superfluid vortex reconnections at non-zero temperatures}

\author{P. Z. Stasiak}
\author{A. Baggaley}
\author{C.F. Barenghi}
\affiliation{School of Mathematics, Statistics and Physics, Newcastle University, Newcastle upon Tyne, NE1 7RU, United Kingdom}

\author{G. Krstulovic}
\affiliation{Universit\'e C\^ote d'Azur, Observatoire de la C\^ote d'Azur, CNRS,Laboratoire Lagrangre, Boulevard de l'Observatoire CS 34229 - F 06304 NICE Cedex 4, France}

\author{L. Galantucci}
\affiliation{Istituto per le Applicazioni del Calcolo ``M. Picone" IAC CNR, Via dei Taurini 19, 00185 Roma, Italy}

\author{\red{W. Guo}}
\author{\red{Y. Xin}}
\author{\red{Y. Alihosseini}}
\affiliation{\red{Department of Mechanical Engineering, FAMU-FSU College of Engineering, Florida State University, Tallahassee, Florida 32310, USA}}
\date{\today}% It is always \today, today,
             %  but any date may be explicitly specified

\begin{abstract}
The minimum separation between reconnecting vortices
in fluids and superfluids obeys a universal scaling law with respect to time.
The pre-reconnection and the post-reconnection prefactors 
of this scaling law are different, a property related to irreversibility.
\red{Using experiments and a} numeric model which fully accounts for the independent dynamics of the superfluid
vortex lines and the thermal normal fluid component, we determine the temperature
dependence of these prefactors. We also \red{numerically} show that each vortex reconnection event
represents a sudden injection of energy in the normal fluid. Finally we argue that
in a turbulent flow, these punctuated energy injections can sustain the normal fluid
in a perturbed state, provided that the density of superfluid vortices is large enough.
\end{abstract}

%\keywords{Suggested keywords}%Use showkeys class option if keyword
   %display desired
\maketitle

\begin{figure*}[t]
	\centering
	\begin{subfigure}[b]{0.24\textwidth}
		\centering
		\includegraphics*[width=\textwidth]{exp-snap-1.png}
	\end{subfigure}
	\begin{subfigure}[b]{0.24\textwidth}
		\centering
		\includegraphics*[width=\textwidth]{exp-snap-2.png}
	\end{subfigure}
    \begin{subfigure}[b]{0.24\textwidth}
		\centering
		\includegraphics*[width=\textwidth]{exp-snap-3.png}
	\end{subfigure}
    \begin{subfigure}[b]{0.24\textwidth}
		\centering
		\includegraphics*[width=\textwidth]{exp-snap-4.png}
	\end{subfigure}
	\hfill
    \vspace{0.5cm}
	\begin{subfigure}[b]{0.24\textwidth}
		\centering
		\includegraphics*[width=\textwidth]{snap-1.png}
	\end{subfigure}
	\begin{subfigure}[b]{0.24\textwidth}
		\centering
		\includegraphics*[width=\textwidth]{snap-2.png}
	\end{subfigure}
    \begin{subfigure}[b]{0.24\textwidth}
		\centering
		\includegraphics*[width=\textwidth]{snap-3.png}
	\end{subfigure}
    \begin{subfigure}[b]{0.24\textwidth}
		\centering
		\includegraphics*[width=\textwidth]{snap-4.png}
	\end{subfigure}
\caption{\red{\emph{Top row:} Snapshots of experiments, details to be added here.}}
\emph{Bottom row:} Oblique collision of two circular vortex rings at different 
(dimensionless) times.
The superfluid vortex lines are represented by red tubes (the radius has been greatly 
exaggerated for visual purposes); the scaled normal fluid enstrophy 
$\bom^2/\bom^2_{max}$ is represented by the blue volume rendering. Here $\bom^2_{max}=50$ in dimensionless units. 
\label{fig:ring-coll-viz}
\end{figure*}

\paragraph*{Introduction.---} Reconnections are the fundamental events that
change the topology of the field lines in fluids and plasmas during their
time evolution. Reconnections
thus determine important physical properties, 
such as mixing and inter-scale energy transfer in fluids \cite{YaoHussainAnnRev2022}, 
or solar flares and tokamak instabilities
in plasmas \cite{Chapman2010}. The nature of reconnections is more clearly
studied if the field lines are concentrated in well-separated
filamentary structures:
vortices in fluids and magnetic flux tubes in plasmas. In superfluid 
helium this concentration is extreme, providing an ideal context:
superfluid vorticity is confined to vortex lines of atomic thickness 
(approximately $a_0 \approx 10^{-10}~\rm m$); a further simplification 
is that, unlike what happens in ordinary fluids, the
circulation of a superfluid vortex  is constrained to the quantized value
$\kappa=h/m=9.97 \times 10^{-8}~\rm m^2/s$, 
where $m$ is the mass of one helium atom and $h$ is Planck's constant. 

It was in this superfluid context that it was theoretically and experimentally recognized
\cite{nazarenko2003,bewley2008,paoletti2010,zuccherQuantumVortexReconnections2012a,villoisUniversalNonuniversalAspects2017a,galantucciCrossoverInteractionDriven2019a}
that reconnections share a universal property irrespective of the initial
condition: the minimum distance between reconnecting 
vortices, $\delta^{\pm}$, scales with time, $t$, according to the form
\begin{equation}
\label{eq:scaling}
	\delta^{\pm}(t) = A^{\pm} (\kappa|t-t_0|)^{1/2},
\end{equation} 
\noindent
where $t_0$ is the reconnection time, and the dimensionless
prefactors $A^-$ and $A^+$ refer respectively to before
($t<t_0$) and after ($t>t_0$) the reconnection. The same scaling law
was then found for reconnections in ordinary viscous fluids 
\cite{yaoSeparationScalingViscous2020}. In the case of a pure
superfluid at temperature $T=0~\rm K$, theoretical work based on
the Gross-Pitaevskii equation (GPE) has shown that
$A^+>A^-$, that is, after the reconnection, vortex lines move away from 
each others faster than in the initial approach; this result has been
related to irreversibility \cite{villoisIrreversibleDynamicsVortex2020},
and is caused by a rarefaction pulse created immediately after the reconnection
\cite{leadbeaterSoundEmissionDue2001b,zuccherQuantumVortexReconnections2012a} which
removes some of the kinetic energy of the vortex configuration.
This acoustic energy loss depends on
the ratio $A^+/A^-$, which in turns depends on the angle of collision
between the vortices \cite{villoisIrreversibleDynamicsVortex2020}. 

However most helium experiments are performed at temperatures $T>1~\rm K$, a regime in which thermal excitations form a fluid called the {\it normal fluid} which provides a viscous route to irreversibility.\red{Modern visualisation techniques rely on active tracer particles to decorate superfluid vortices \cite{paoletti2008velocity,bewley2008,guo2014visualization}. Numerous studies have provided insight into the post-reconnection dynamics and the prefactor $A^{+}$, but much less is known about $A^-$ from experiments due to statistical likelihood of observation in the plane of view.}

The aim of this Letter is to investigate the role played by the normal fluid in the reconnection dynamics. In particular, given the
temperature dependence of the normal fluid's properties, we study \red{experimentally and numerically} the temperature
dependence of the prefactors $A^+$ and $A^-$ and \red{numerically investigate} the energy injected in the normal fluid. 

To achieve this aim we need a more powerful model than the GPE to account not only for the dynamics of the
superfluid vortices, but also for the dynamics of the normal fluid. 
%The presence of the second, viscous normal fluid leads to a two-fluid 
%vortex reconnection.
We show that at non-zero temperatures Eq.~(\ref{eq:scaling}) and the
relation $A^+>A^-$ hold true, in agreement with experiments, revealing, for the 
first time, a temperature dependence of $A^+/A^-$. In addition, we
show that a vortex
reconnection represents an unusual kind of
punctuated energy injection into the normal fluid which acts alongside
the well-known (continual) friction.
When applied to superfluid turbulence, this last result implies that,
if the vortex line density (hence the frequency of reconnections) 
is large enough, vortex
reconnections can maintain the normal fluid in a perturbed state.
%%%%%%%%%%%%%%%%%%%%%%%%%%%%%%%%%%%%%%%%%%%%%%%%%%%%%%%%%%%
\red{
\paragraph*{Experimental Method.---} Add details of the experimental method here.
}
%%%%%%%%%%%%%%%%%%%%%%%%%%%%%%%%%%%%%%%%%%%%%%%%%%%%%%%%%%%
\paragraph*{Numerical Method.---}We follow the approach of Schwarz \cite{schwarz1988} which exploits
the vast separation of length scales between the vortex core $a_0$ and any
other relevant distance, in particular the average distance between vortices,
$\ell$, in the case of turbulence. Vortex lines are described
as space curves $\s(\xi,t)$ where $\xi$ is arclength. The equation of motion 
of the vortex lines is
\begin{equation}
\label{eq:s}
	\dot{\s}(\xi,t) = \v_s + \frac{\beta}{(1+\beta)}\left[\v_{ns}\cdot \s'\right]\s' + \beta\s'\times\v_{ns}+\beta'\s'\times\left[\s'\times \v_{ns}\right],
\end{equation}

\noindent
where $\dot{\s}=\partial\s/\partial t$, $\s'=\partial\s/\partial \xi$ 
is the unit tangent vector, 
$\v_n$ and $\v_s$ are the normal fluid and superfluid velocities at $\s$,
$\v_{ns}=\v_n - \v_s$, 
and $\beta$, $\beta'$ are temperature and Reynolds number dependent 
mutual friction coefficients \cite{galantucciNewSelfconsistentApproach2020b}. Superfluid vortices are coupled to a classical description of the 
incompressible ($\nabla\cdot\v_n=0$) normal fluid via the mutual 
friction force $\mathbf{F}_{ns}$, \red{an internal injection in the Navier-Stokes model}

\begin{equation}
\label{eq:vn}
	\frac{\partial \v_n}{\partial t} + (\v_n\cdot\nabla)\v_n = 
        -\frac{1}{\rho} \nabla p + \nu_n\nabla^2\v_n + \frac{\mathbf{F}_{ns}}{\rho_n},
\end{equation}

\noindent
where $\rho=\rho_n + \rho_s$ is the total density, $\rho_n$ and 
$\rho_s$ are the normal fluid and superfluid densities, $p$ is the pressure, 
and $\nu_n$ is the kinematic viscosity of the normal fluid. 
Equations~(\ref{eq:s}) and (\ref{eq:vn})
are solved in dimensionless form using the length unit 
$\lambda=1.59 \times 10^{-4}~\rm m$ and the time unit $\tau$ defined next, 
see also \cite{SeeSupplementaryMaterials} 
for details.
The algorithm for vortex reconnections is standard
\cite{baggaleySensitivityVortexFilament2012a}. We consider two distinct initial vortex configurations
at three temperatures $T=0~\rm K$, $1.9~\rm K$ and $2.1~\rm K$
corresponding to the superfluid fractions $\rho_s/\rho=100 \%$,
$58 \%$ and $26 \%$. To make the equations dimensionless,
we use the time units $\tau=0.183~\rm s$ at $T=0~\rm K$
and $1.9~\rm K$, and $\tau=0.242~\rm s$ at $T=2.1~\rm K$. 
All configurations lead to a vortex reconnection.
The first configuration consists of two vortex rings 
of (dimensionless) radius $R\approx 1$ in a tent-like shape
which collide obliquely 
making an initial angle $\alpha$ with the vertical direction, 
as shown in Fig.~\ref{fig:ring-coll-viz}, and,
schematically, in Fig.~\ref{fig:prefactors}.   
By changing the parameter $\alpha$, we create a sample of 12 realizations
at each temperature (again, see the Supplementary Material 
\cite{SeeSupplementaryMaterials} for details).
%We take 12 realizations of $\alpha$, such that 
%$\alpha\in\lbrace i\pi/13|i=1,\cdots,12\rbrace$. 
%The supplementary material 
%gives more details of our model and these configurations
%\cite{SeeSupplementaryMaterials}. 
The second configuration is the Hopf link, shown schematically in Fig.~\ref{fig:prefactors}. It consists of two perpendicular linked rings of radius $R\approx1$ with an offset in the $xy$-plane.
By changing the offset, we create a sample of 49 reconnections
at each temperature, as described in the Supplementary Material
\cite{SeeSupplementaryMaterials}.
%defined by parameters $\Delta l_x$ and $\Delta l_y$ 
%The offsets are chosen so that 
%$(\Delta l_x, \Delta l _y) \in \lbrace(0.125i,0.125j)|i,j=-3,\cdots,3 \rbrace$,
%providing us with a total of 49 reconnections for each temperature. 
In all cases, normal fluid structures
in the form of rings \cite{kivotides-barenghi-samuels-2000} are 
initially superimposed to match the vortex lines, eliminating the transient 
phase of generating these normal fluid structures.


\paragraph*{Scaling law. ---}\red{In the experiment, two reconnections were observed where both $A^+$ and $A^-$ could be identified and calculated, at $T=1.65K$ and $T=2K$, plotted as orange triangles in Fig.~\ref{fig:prefactors}. The pre-reconnection factor $A^-$ lies within the 0.4-0.6 range, consistent with the results of the numerics, and a clear temperature effect between a superfluid component majority and normal fluid component majority.}
In the case of the Hopf link 
%the interaction between the two rings leads to a reconnection.  
we have performed 147 simulations 
(49 across 3 temperatures) as shown in Fig.~\ref{fig:minimum-distance} and
verified Eq.~(\ref{eq:scaling}) for the minimum distance $\delta^{\pm}$. 
The prefactors $A^{\pm}$ have been computed in the shaded region 
of the figure. In the pre-reconnection regime ($t<t_0$) we observe
a clear segregation of the values of $A^-$ due to temperature: 
the minimum distance grows more rapidly with time if the temperature is
lowered. 
In stark contrast, there is no memory of the temperature in the
post-reconnection regime ($t>t_0$). 
%\red{The lack of a temperature dependence for $A^+$ suggests
%that the normal fluid plays only a minor role in the
%dynamics of the post-reconnection regime.


At $T=0~{\rm K}$ our calculations for superfluid helium 
(black symbols in Fig.~\ref{fig:prefactors}) are in good agreement 
with previous results obtained with the GPE \cite{villoisIrreversibleDynamicsVortex2020}
(green diamonds), showing irreversible dynamics. In addition, the computed values of $A^-\approx 0.4$-$0.6$ at $T=0~{\rm K}$
are consistent with analytical calculations \cite{boue-etal-2013,rica-2019}. 
At non-zero temperatures, our results confirm the irreversibility of vortex reconnections 
observed at $T=0$ as $A^+$ is always larger than $A^-$. Importantly, this asymmetry 
is recovered in all our simulations, regardless of their initial condition.
%Our results confirm the irreversibility of vortex reconnections 
%\red{($A^+>A^-$)} at non-zero temperatures. 
%As shown in Fig.~\ref{fig:prefactors}, \red{our} $T=0~{\rm K}$ calculation 
%for superfluid helium is in good agreement with the GPE model for
%a condensate at $T=0$, which yields $A^-\approx 0.4$-$0.6$ 
%for both initial conditions \red{[ADD REF]}. 
The same asymmetry between
$A^+$ and $A^-$ at non-zero temperatures has been observed for reconnections in finite-temperature
Bose-Einstein condensatates \cite{allen2014}, although in this work the system
is not homogeneous (the condensate is confined by a harmonic trap) and the
thermal component is a ballistc gas, not a viscous fluid.
Recent investigations of vortex reconnections in classical viscous fluids %the classical Navier-Stokes equation
\cite{yaoSeparationScalingViscous2020} also display
a clear $1/2$ power-law scaling for the minimum distance
with $A^- \approx 0.3$-$0.4$, which again shows good agreement with our
results. The scaling law (Eq.~\ref{eq:scaling}) and the value of $A^-$ hence appear 
to have a universal character in vortex reconnections, independently of the nature
of the fluid, classical or quantum, and temperature.


%\paragraph*{\red{Energy injection. ---}}

\begin{figure}[t]
	\centering
	\begin{subfigure}[b]{0.45\textwidth}
		\centering
		\includegraphics*[width=\textwidth]{min-delta.pdf}
		\caption{}
		\label{fig:minimum-distance}
	\end{subfigure}
	\begin{subfigure}[b]{0.45\textwidth}
		\centering
		\includegraphics*[width=\textwidth]{prefactors-with-exp.pdf}
		\caption{}
		\label{fig:prefactors}
	\end{subfigure}
\caption{\emph{(a)}: Time evolution of the (dimensionless)
minimum distance squared $\delta^2$ plotted versus (dimensionless) $\kappa (t-t_0)$ for
the Hopf link reconnections at $T=0K,1.9K$ and $2.1K$ (black, blue and red
respectively). The grey shaded areas
are the regions used to estimate the prefactors $A^{\pm}$. 
\emph{Inset:} Values of the separation prefactor $A^+$ and approach prefactors $A^-$. 
The dashed line corresponds to $A^+ = A^-$
\emph{(b)}: Comparison of all prefactors: Hopf links (\emph{HL}, circles), 
ring collisions (\emph{RC}, stars with yellow outline), GPE-data 
from Villois \etal \cite{villoisIrreversibleDynamicsVortex2020} (green diamonds) and experimental results from this study (orange triangles). 
The shaded areas associated with each colour represent the convex hull of errors 
for each temperature. Schematic rendering of initial conditions are included.}
\end{figure}

\paragraph*{Energy injection. ---}
The normal fluid impacts the dynamics of reconnecting superfluid vortices via the temperature dependent mutual friction coefficients. Conversely, the motion of superfluid vortices involved in the reconnection process influence, significantly, the dynamics of the normal fluid. Fig.~\ref{fig:energy-evol} indeed shows that the normal fluid energy, $E_n$,
suddenly increases at the reconnection time by an amount ($\approx 5\%$)
which is smaller but comparable to the continuous energy increase as vortex
lines approach each other. Indeed the
curvature $\zeta=|\s''|$ of the vortex line 
spikes at $t=t_0$ when
the reconnection cusp is created, and, in the first approximation \cite{galantucci-krstulovic-etal-2023},
the magnitude of the energy injected in the normal fluid per unit time $I$
is proportional to the strength of the mutual friction force $\mathbf{F}_{ns}$ which scales as
$|\mathbf{F}_{ns}(\s)|\propto|\dot{\s}-\v_n|\propto|\dot{\s}|\propto\zeta$. This sudden
transfer of energy
\cite{stasiakCrossComponentEnergyTransfer2024} from the superfluid
vortex configuration to the normal fluid is the origin of the 
small scale normal fluid enstrophy structures
which are visible in Fig.~\ref{fig:ring-coll-viz}.


%The prefactor ratio $A^+/A^-$ has been shown in recent literature 
%\cite{villoisIrreversibleDynamicsVortex2020,villoisUniversalNonuniversalAspects2017a,promentMatchingTheoryCharacterize2020} 
%to be of importance in describing fundamental properties of reconnections 
%using the Gross-Pitaevskii (GP) equation. Due to the \emph{ad-hoc} 
%nature of vortex reconnections in our helium model, it is not trivial 
%to derive a linear theory in the limit $\delta^{\pm}\rightarrow0$. 


\begin{figure}
	\centering
	\includegraphics*[width=0.45\textwidth]{energy-evolution.pdf}
	\caption{Normal fluid kinetic energy $E_n$ scaled by $E_n^0$ (the 
kinetic energy at $t=t_0$), plotted versus (dimensionless)
$\kappa (t-t_0)$ for the Hopf link reconnections. Black diamonds represent the simulations 
with minimum and maximum prefactor ratios $A^+/A^-$ at $T=1.9~\rm K$ and 
$T=2.1~\rm K$ respectively.}
	\label{fig:energy-evol}
\end{figure}

The total energy injected into the normal fluid by the reconnection,
$\Delta E_n$, which hereafter
we refer to as the energy jump, is defined as

\begin{equation}
	\Delta E_n = \max{\left[E_n(t>t_0)\right]} - E_n^0,
\end{equation} 

\noindent
where $E_n^0=E_n(t_0)$ is the normal fluid kinetic energy at $t=t_0$.
Normalized energy jumps are plotted in
Fig.~\ref{fig:energy-jumps} as a function of
the ratio $A^+/A^-$. Here, we observe that the larger $A^+/A^-$ is 
(i.e the more the reconnection is anti-parallel \cite{villoisIrreversibleDynamicsVortex2020}),
the smaller the normal fluid excitation is.

The emission of the sound pulse at the vortex reconnection 
\cite{leadbeaterSoundEmissionDue2001b} which is typical of the GPE model
is absent in our incompressible hydrodynamic approach. To model
this effect, the change of vortex length, $\Delta L$, created by the
vortex reconnection algorithm is always negative by construction
\cite{baggaleySensitivityVortexFilament2012a},
because, in the local induction approximation to the Biot-Savart law, 
the superfluid incompressible kinetic energy, $E_s$, is proportional to the vortex length,
$L$. Such procedure ensures that at $T=0~{\rm K}$ when a reconnection occurs
$\Delta E_s\propto \Delta L < 0$. %Therefore, at $T=0~{\rm K}$, in the absence of any dissipative 
%normal fluid, $\Delta E_s\propto \Delta L$. 
Consequentially, in the absence of any dissipative normal fluid,
the superfluid energy 
$E_s$ that would be transferred to the sound pulse, normalized with its value $E_s^0$ at 
reconnection, is $-\Delta L/L_0$., 
%should exhibit the same characteristic behavior as the normal fluid energy 
%which is injected, $\Delta E_n/E_n^0$. 
If these normalized energy jumps (black diamonds
in Fig.~\ref{fig:energy-jumps}) are compared to the results obtained with the 
compressible GPE \cite{villoisIrreversibleDynamicsVortex2020} (purple squares)
we find a good agreement, confirming that the model we employ, 
%despite the ad-hoc algorithm employed to take into account the reconnection events
is suitable for the investigation of the feature of single reconnection events.
%Indeed, the black diamonds
%in Fig.~\ref{fig:energy-jumps} shows that our $T=0~{\rm K}$ simulations are
%in good agreement with the GPE results (purple squares).



\paragraph*{Implications for turbulence. ---}
Our numerical results have implications for our understanding of 
quantum turbulence \cite{BSS2023}.
A fully developed turbulent tangle of vortices is
characterized by its vortex line density $\mathcal{L}$ (vortex length
per unit volume); the frequency 
of vortex reconnections per unit volume is 
${f=(\kappa/6\pi)\mathcal{L}^{5/2}\ln(\mathcal{L}^{-1/2}/a_0)}$
\cite{barenghi2004}. From Fig.~\ref{fig:energy-evol} we estimate the 
normal fluid reconnection relaxation time $\tau_n$ as the time 
after reconnection at which the normal fluid energy $E_n/E_0$ has decayed
to the pre-reconnection level: in our dimensionless units, $\kappa \tau_n \approx 0.25$. 
Using this timescale, we estimate that
the average vortex line density that is required to sustain the normal fluid 
in a perturbed state via frequent vortex reconnections is approximately
$\mathcal{L} \approx 10^7$ to $10^8\mathrm{m}^{-2}$. Experiments in $^4$He
\cite{schwarz1981,milliken1982,roche2008,roche2007,Babuin2014} and in $^3$He
\cite{bradley2006} can achieve vortex line densities much larger than this.


\begin{figure}
	\centering
	\includegraphics*[width=0.48\textwidth]{energy-jump.pdf}
	\caption{Normalized energy jumps $\Delta E_n/E_n^0$ for Hopf link 
reconnections.The solid black diamonds are the normalized change in line length $\Delta L/L_0$ in the
$T=0~\rm K$ case. Blue and red circle correspond to $T=1.9K$ and $T=2.1K$,respectively. The purple squares are from GPE simulations of
Villois et al. \cite{villoisIrreversibleDynamicsVortex2020}.}
	\label{fig:energy-jumps}
\end{figure}

\paragraph*{Conclusions.---} \red{We have conducted an experiment using active particle tracers} and a statistical numerical study
of vortex reconnections in a wide range of temperatures using a model of $^4$He which accounts for the coupled dynamics of superfluid and normal fluid components.
We have verified the scaling law of the minimum vortex distance 
$\delta^{\pm}=A^{\pm} (\kappa |t-t_0|)^{1/2}$ and found that the approach prefactor $A^-$ has a clear temperature dependence independent of the geometry \red{in both experiments and numerics}, in contrast to the
separation prefactor $A^+$. The prefactors are in good agreement
with GPE simulations \cite{villoisIrreversibleDynamicsVortex2020,allen2014} 
and classical fluid reconnections \cite{yaoSeparationScalingViscous2020}
revealing that vortex reconnections display a universal behaviour regardless of the nature
of the fluid (classical or quantum) and of temeperature. It is worth noting that the behaviour, as a function of $A^+/A^-$, of 
the energy injected in the normal fluid (at $T>0$) and of the energy transferred to sound (at $T=0$)
\cite{villoisIrreversibleDynamicsVortex2020,leadbeaterSoundEmissionDue2001b} is dissimilar: the former decreases as $A^+/A^-$
increases, the latter the opposite. This likely arises from the distinct physics governing the loss of superfluid 
incompressible kinetic energy: mutual friction at $T>0$, quantum pressure at $T=0$.
We have also found that a reconnection event suddenly injects an amount of energy 
into the normal fluid which is comparable to the energy transferred by friction
during the vortex approach. Applying these results to turbulence, we have
compared the decay time of the normal fluid structures created by a
reconnection to the frequency of reconnections in a vortex tangle, and argued
that, if the vortex line density is large enough, these punctuated
energy injections should sustain the normal fluid in a perturbed state.

\bibliography{references3}% Produces the bibliography via BibTeX.

% % ****** Start of file apssamp.tex ******
%
%   This file is part of the APS files in the REVTeX 4.2 distribution.
%   Version 4.2a of REVTeX, December 2014
%
%   Copyright (c) 2014 The American Physical Society.
%
%   See the REVTeX 4 README file for restrictions and more information.
%
% TeX'ing this file requires that you have AMS-LaTeX 2.0 installed
% as well as the rest of the prerequisites for REVTeX 4.2
%
% See the REVTeX 4 README file
% It also requires running BibTeX. The commands are as follows:
%
%  1)  latex apssamp.tex
%  2)  bibtex apssamp
%  3)  latex apssamp.tex
%  4)  latex apssamp.tex
%
\documentclass[%
 reprint,
% superscriptaddress,
%groupedaddress,
%unsortedaddress,
%runinaddress,
%frontmatterverbose, 
%preprint,
%preprintnumbers,
%nofootinbib,
%nobibnotes,
%bibnotes,
 amsmath,amssymb,
 aps,
 prl,
%pra,
% prb,
% rmp,
%prstab,
%prstper,
%floatfix,
]{revtex4-2}

\usepackage{graphicx}% Include figure files
\usepackage{dcolumn}% Align table columns on decimal point
\usepackage{bm}% bold math
\usepackage{blindtext}
\usepackage{float}
\usepackage{caption}
\usepackage{cleveref}
\usepackage{subcaption}
\usepackage{xcolor}
%\usepackage{hyperref}% add hypertext capabilities
%\usepackage[mathlines]{lineno}% Enable numbering of text and display math
%\linenumbers\relax % Commence numbering lines

%\usepackage[showframe,%Uncomment any one of the following lines to test 
%%scale=0.7, marginratio={1:1, 2:3}, ignoreall,% default settings
%%text={7in,10in},centering,
%%margin=1.5in,
%%total={6.5in,8.75in}, top=1.2in, left=0.9in, includefoot,
%%height=10in,a5paper,hmargin={3cm,0.8in},
%]{geometry}


\newcommand{\etal}{{\it et al.}~}
\newcommand{\bom}{\boldsymbol{\omega}}

% \newcommand{\sd}[3][\null]{\ensuremath{\dfrac{\d^{#1} #2}{\d #3^{#1}}}}%Standard derivative 
% \newcommand{\pd}[3][\null]{\ensuremath{\dfrac{\partial^{#1} #2}{\partial #3^{#1}}}}%Partial derivative
% \newcommand{\matd}[2][\null]{\ensuremath{\dfrac{\mathrm{D}^{#1} #2}{\mathrm{D} t^{#1}}}} %Material derivative

\def \s{\mathbf{s}}
\def \v{\mathbf{v}}
\def \x{\mathbf{x}}
\def \r{\mathbf{r}}
\def \k{\mathbf{k}}

\def \cm{\mathrm{cm}}
\def \cms{\mathrm{cm/s}}
\def \sec{\mathrm{s}}
\def \K{\mathrm{K}}

\def\red#1{\textcolor{red}{#1}}
\def\blue#1{\textcolor{blue}{#1}}
%%%%%%%%%%%%%%%%%%%%%%%%%%%%%%%%%



\begin{document}

\preprint{APS/123-QED}

\title{Superfluid vortex reconnections at non-zero temperatures}

\author{P. Z. Stasiak}
\author{A. Baggaley}
\author{C.F. Barenghi}
\affiliation{School of Mathematics, Statistics and Physics, Newcastle University, Newcastle upon Tyne, NE1 7RU, United Kingdom}

\author{G. Krstulovic}
\affiliation{Universit\'e C\^ote d'Azur, Observatoire de la C\^ote d'Azur, CNRS,Laboratoire Lagrangre, Boulevard de l'Observatoire CS 34229 - F 06304 NICE Cedex 4, France}

\author{L. Galantucci}
\affiliation{Istituto per le Applicazioni del Calcolo ``M. Picone" IAC CNR, Via dei Taurini 19, 00185 Roma, Italy}

\author{\red{W. Guo}}
\author{\red{Y. Xin}}
\author{\red{Y. Alihosseini}}
\affiliation{\red{Department of Mechanical Engineering, FAMU-FSU College of Engineering, Florida State University, Tallahassee, Florida 32310, USA}}
\date{\today}% It is always \today, today,
             %  but any date may be explicitly specified

\begin{abstract}
The minimum separation between reconnecting vortices
in fluids and superfluids obeys a universal scaling law with respect to time.
The pre-reconnection and the post-reconnection prefactors 
of this scaling law are different, a property related to irreversibility.
\red{Using experiments and a} numeric model which fully accounts for the independent dynamics of the superfluid
vortex lines and the thermal normal fluid component, we determine the temperature
dependence of these prefactors. We also \red{numerically} show that each vortex reconnection event
represents a sudden injection of energy in the normal fluid. Finally we argue that
in a turbulent flow, these punctuated energy injections can sustain the normal fluid
in a perturbed state, provided that the density of superfluid vortices is large enough.
\end{abstract}

%\keywords{Suggested keywords}%Use showkeys class option if keyword
   %display desired
\maketitle

\begin{figure*}[t]
	\centering
	\begin{subfigure}[b]{0.24\textwidth}
		\centering
		\includegraphics*[width=\textwidth]{exp-snap-1.png}
	\end{subfigure}
	\begin{subfigure}[b]{0.24\textwidth}
		\centering
		\includegraphics*[width=\textwidth]{exp-snap-2.png}
	\end{subfigure}
    \begin{subfigure}[b]{0.24\textwidth}
		\centering
		\includegraphics*[width=\textwidth]{exp-snap-3.png}
	\end{subfigure}
    \begin{subfigure}[b]{0.24\textwidth}
		\centering
		\includegraphics*[width=\textwidth]{exp-snap-4.png}
	\end{subfigure}
	\hfill
    \vspace{0.5cm}
	\begin{subfigure}[b]{0.24\textwidth}
		\centering
		\includegraphics*[width=\textwidth]{snap-1.png}
	\end{subfigure}
	\begin{subfigure}[b]{0.24\textwidth}
		\centering
		\includegraphics*[width=\textwidth]{snap-2.png}
	\end{subfigure}
    \begin{subfigure}[b]{0.24\textwidth}
		\centering
		\includegraphics*[width=\textwidth]{snap-3.png}
	\end{subfigure}
    \begin{subfigure}[b]{0.24\textwidth}
		\centering
		\includegraphics*[width=\textwidth]{snap-4.png}
	\end{subfigure}
\caption{\red{\emph{Top row:} Snapshots of experiments, details to be added here.}}
\emph{Bottom row:} Oblique collision of two circular vortex rings at different 
(dimensionless) times.
The superfluid vortex lines are represented by red tubes (the radius has been greatly 
exaggerated for visual purposes); the scaled normal fluid enstrophy 
$\bom^2/\bom^2_{max}$ is represented by the blue volume rendering. Here $\bom^2_{max}=50$ in dimensionless units. 
\label{fig:ring-coll-viz}
\end{figure*}

\paragraph*{Introduction.---} Reconnections are the fundamental events that
change the topology of the field lines in fluids and plasmas during their
time evolution. Reconnections
thus determine important physical properties, 
such as mixing and inter-scale energy transfer in fluids \cite{YaoHussainAnnRev2022}, 
or solar flares and tokamak instabilities
in plasmas \cite{Chapman2010}. The nature of reconnections is more clearly
studied if the field lines are concentrated in well-separated
filamentary structures:
vortices in fluids and magnetic flux tubes in plasmas. In superfluid 
helium this concentration is extreme, providing an ideal context:
superfluid vorticity is confined to vortex lines of atomic thickness 
(approximately $a_0 \approx 10^{-10}~\rm m$); a further simplification 
is that, unlike what happens in ordinary fluids, the
circulation of a superfluid vortex  is constrained to the quantized value
$\kappa=h/m=9.97 \times 10^{-8}~\rm m^2/s$, 
where $m$ is the mass of one helium atom and $h$ is Planck's constant. 

It was in this superfluid context that it was theoretically and experimentally recognized
\cite{nazarenko2003,bewley2008,paoletti2010,zuccherQuantumVortexReconnections2012a,villoisUniversalNonuniversalAspects2017a,galantucciCrossoverInteractionDriven2019a}
that reconnections share a universal property irrespective of the initial
condition: the minimum distance between reconnecting 
vortices, $\delta^{\pm}$, scales with time, $t$, according to the form
\begin{equation}
\label{eq:scaling}
	\delta^{\pm}(t) = A^{\pm} (\kappa|t-t_0|)^{1/2},
\end{equation} 
\noindent
where $t_0$ is the reconnection time, and the dimensionless
prefactors $A^-$ and $A^+$ refer respectively to before
($t<t_0$) and after ($t>t_0$) the reconnection. The same scaling law
was then found for reconnections in ordinary viscous fluids 
\cite{yaoSeparationScalingViscous2020}. In the case of a pure
superfluid at temperature $T=0~\rm K$, theoretical work based on
the Gross-Pitaevskii equation (GPE) has shown that
$A^+>A^-$, that is, after the reconnection, vortex lines move away from 
each others faster than in the initial approach; this result has been
related to irreversibility \cite{villoisIrreversibleDynamicsVortex2020},
and is caused by a rarefaction pulse created immediately after the reconnection
\cite{leadbeaterSoundEmissionDue2001b,zuccherQuantumVortexReconnections2012a} which
removes some of the kinetic energy of the vortex configuration.
This acoustic energy loss depends on
the ratio $A^+/A^-$, which in turns depends on the angle of collision
between the vortices \cite{villoisIrreversibleDynamicsVortex2020}. 

However most helium experiments are performed at temperatures $T>1~\rm K$, a regime in which thermal excitations form a fluid called the {\it normal fluid} which provides a viscous route to irreversibility.\red{Modern visualisation techniques rely on active tracer particles to decorate superfluid vortices \cite{paoletti2008velocity,bewley2008,guo2014visualization}. Numerous studies have provided insight into the post-reconnection dynamics and the prefactor $A^{+}$, but much less is known about $A^-$ from experiments due to statistical likelihood of observation in the plane of view.}

The aim of this Letter is to investigate the role played by the normal fluid in the reconnection dynamics. In particular, given the
temperature dependence of the normal fluid's properties, we study \red{experimentally and numerically} the temperature
dependence of the prefactors $A^+$ and $A^-$ and \red{numerically investigate} the energy injected in the normal fluid. 

To achieve this aim we need a more powerful model than the GPE to account not only for the dynamics of the
superfluid vortices, but also for the dynamics of the normal fluid. 
%The presence of the second, viscous normal fluid leads to a two-fluid 
%vortex reconnection.
We show that at non-zero temperatures Eq.~(\ref{eq:scaling}) and the
relation $A^+>A^-$ hold true, in agreement with experiments, revealing, for the 
first time, a temperature dependence of $A^+/A^-$. In addition, we
show that a vortex
reconnection represents an unusual kind of
punctuated energy injection into the normal fluid which acts alongside
the well-known (continual) friction.
When applied to superfluid turbulence, this last result implies that,
if the vortex line density (hence the frequency of reconnections) 
is large enough, vortex
reconnections can maintain the normal fluid in a perturbed state.
%%%%%%%%%%%%%%%%%%%%%%%%%%%%%%%%%%%%%%%%%%%%%%%%%%%%%%%%%%%
\red{
\paragraph*{Experimental Method.---} Add details of the experimental method here.
}
%%%%%%%%%%%%%%%%%%%%%%%%%%%%%%%%%%%%%%%%%%%%%%%%%%%%%%%%%%%
\paragraph*{Numerical Method.---}We follow the approach of Schwarz \cite{schwarz1988} which exploits
the vast separation of length scales between the vortex core $a_0$ and any
other relevant distance, in particular the average distance between vortices,
$\ell$, in the case of turbulence. Vortex lines are described
as space curves $\s(\xi,t)$ where $\xi$ is arclength. The equation of motion 
of the vortex lines is
\begin{equation}
\label{eq:s}
	\dot{\s}(\xi,t) = \v_s + \frac{\beta}{(1+\beta)}\left[\v_{ns}\cdot \s'\right]\s' + \beta\s'\times\v_{ns}+\beta'\s'\times\left[\s'\times \v_{ns}\right],
\end{equation}

\noindent
where $\dot{\s}=\partial\s/\partial t$, $\s'=\partial\s/\partial \xi$ 
is the unit tangent vector, 
$\v_n$ and $\v_s$ are the normal fluid and superfluid velocities at $\s$,
$\v_{ns}=\v_n - \v_s$, 
and $\beta$, $\beta'$ are temperature and Reynolds number dependent 
mutual friction coefficients \cite{galantucciNewSelfconsistentApproach2020b}. Superfluid vortices are coupled to a classical description of the 
incompressible ($\nabla\cdot\v_n=0$) normal fluid via the mutual 
friction force $\mathbf{F}_{ns}$, \red{an internal injection in the Navier-Stokes model}

\begin{equation}
\label{eq:vn}
	\frac{\partial \v_n}{\partial t} + (\v_n\cdot\nabla)\v_n = 
        -\frac{1}{\rho} \nabla p + \nu_n\nabla^2\v_n + \frac{\mathbf{F}_{ns}}{\rho_n},
\end{equation}

\noindent
where $\rho=\rho_n + \rho_s$ is the total density, $\rho_n$ and 
$\rho_s$ are the normal fluid and superfluid densities, $p$ is the pressure, 
and $\nu_n$ is the kinematic viscosity of the normal fluid. 
Equations~(\ref{eq:s}) and (\ref{eq:vn})
are solved in dimensionless form using the length unit 
$\lambda=1.59 \times 10^{-4}~\rm m$ and the time unit $\tau$ defined next, 
see also \cite{SeeSupplementaryMaterials} 
for details.
The algorithm for vortex reconnections is standard
\cite{baggaleySensitivityVortexFilament2012a}. We consider two distinct initial vortex configurations
at three temperatures $T=0~\rm K$, $1.9~\rm K$ and $2.1~\rm K$
corresponding to the superfluid fractions $\rho_s/\rho=100 \%$,
$58 \%$ and $26 \%$. To make the equations dimensionless,
we use the time units $\tau=0.183~\rm s$ at $T=0~\rm K$
and $1.9~\rm K$, and $\tau=0.242~\rm s$ at $T=2.1~\rm K$. 
All configurations lead to a vortex reconnection.
The first configuration consists of two vortex rings 
of (dimensionless) radius $R\approx 1$ in a tent-like shape
which collide obliquely 
making an initial angle $\alpha$ with the vertical direction, 
as shown in Fig.~\ref{fig:ring-coll-viz}, and,
schematically, in Fig.~\ref{fig:prefactors}.   
By changing the parameter $\alpha$, we create a sample of 12 realizations
at each temperature (again, see the Supplementary Material 
\cite{SeeSupplementaryMaterials} for details).
%We take 12 realizations of $\alpha$, such that 
%$\alpha\in\lbrace i\pi/13|i=1,\cdots,12\rbrace$. 
%The supplementary material 
%gives more details of our model and these configurations
%\cite{SeeSupplementaryMaterials}. 
The second configuration is the Hopf link, shown schematically in Fig.~\ref{fig:prefactors}. It consists of two perpendicular linked rings of radius $R\approx1$ with an offset in the $xy$-plane.
By changing the offset, we create a sample of 49 reconnections
at each temperature, as described in the Supplementary Material
\cite{SeeSupplementaryMaterials}.
%defined by parameters $\Delta l_x$ and $\Delta l_y$ 
%The offsets are chosen so that 
%$(\Delta l_x, \Delta l _y) \in \lbrace(0.125i,0.125j)|i,j=-3,\cdots,3 \rbrace$,
%providing us with a total of 49 reconnections for each temperature. 
In all cases, normal fluid structures
in the form of rings \cite{kivotides-barenghi-samuels-2000} are 
initially superimposed to match the vortex lines, eliminating the transient 
phase of generating these normal fluid structures.


\paragraph*{Scaling law. ---}\red{In the experiment, two reconnections were observed where both $A^+$ and $A^-$ could be identified and calculated, at $T=1.65K$ and $T=2K$, plotted as orange triangles in Fig.~\ref{fig:prefactors}. The pre-reconnection factor $A^-$ lies within the 0.4-0.6 range, consistent with the results of the numerics, and a clear temperature effect between a superfluid component majority and normal fluid component majority.}
In the case of the Hopf link 
%the interaction between the two rings leads to a reconnection.  
we have performed 147 simulations 
(49 across 3 temperatures) as shown in Fig.~\ref{fig:minimum-distance} and
verified Eq.~(\ref{eq:scaling}) for the minimum distance $\delta^{\pm}$. 
The prefactors $A^{\pm}$ have been computed in the shaded region 
of the figure. In the pre-reconnection regime ($t<t_0$) we observe
a clear segregation of the values of $A^-$ due to temperature: 
the minimum distance grows more rapidly with time if the temperature is
lowered. 
In stark contrast, there is no memory of the temperature in the
post-reconnection regime ($t>t_0$). 
%\red{The lack of a temperature dependence for $A^+$ suggests
%that the normal fluid plays only a minor role in the
%dynamics of the post-reconnection regime.


At $T=0~{\rm K}$ our calculations for superfluid helium 
(black symbols in Fig.~\ref{fig:prefactors}) are in good agreement 
with previous results obtained with the GPE \cite{villoisIrreversibleDynamicsVortex2020}
(green diamonds), showing irreversible dynamics. In addition, the computed values of $A^-\approx 0.4$-$0.6$ at $T=0~{\rm K}$
are consistent with analytical calculations \cite{boue-etal-2013,rica-2019}. 
At non-zero temperatures, our results confirm the irreversibility of vortex reconnections 
observed at $T=0$ as $A^+$ is always larger than $A^-$. Importantly, this asymmetry 
is recovered in all our simulations, regardless of their initial condition.
%Our results confirm the irreversibility of vortex reconnections 
%\red{($A^+>A^-$)} at non-zero temperatures. 
%As shown in Fig.~\ref{fig:prefactors}, \red{our} $T=0~{\rm K}$ calculation 
%for superfluid helium is in good agreement with the GPE model for
%a condensate at $T=0$, which yields $A^-\approx 0.4$-$0.6$ 
%for both initial conditions \red{[ADD REF]}. 
The same asymmetry between
$A^+$ and $A^-$ at non-zero temperatures has been observed for reconnections in finite-temperature
Bose-Einstein condensatates \cite{allen2014}, although in this work the system
is not homogeneous (the condensate is confined by a harmonic trap) and the
thermal component is a ballistc gas, not a viscous fluid.
Recent investigations of vortex reconnections in classical viscous fluids %the classical Navier-Stokes equation
\cite{yaoSeparationScalingViscous2020} also display
a clear $1/2$ power-law scaling for the minimum distance
with $A^- \approx 0.3$-$0.4$, which again shows good agreement with our
results. The scaling law (Eq.~\ref{eq:scaling}) and the value of $A^-$ hence appear 
to have a universal character in vortex reconnections, independently of the nature
of the fluid, classical or quantum, and temperature.


%\paragraph*{\red{Energy injection. ---}}

\begin{figure}[t]
	\centering
	\begin{subfigure}[b]{0.45\textwidth}
		\centering
		\includegraphics*[width=\textwidth]{min-delta.pdf}
		\caption{}
		\label{fig:minimum-distance}
	\end{subfigure}
	\begin{subfigure}[b]{0.45\textwidth}
		\centering
		\includegraphics*[width=\textwidth]{prefactors-with-exp.pdf}
		\caption{}
		\label{fig:prefactors}
	\end{subfigure}
\caption{\emph{(a)}: Time evolution of the (dimensionless)
minimum distance squared $\delta^2$ plotted versus (dimensionless) $\kappa (t-t_0)$ for
the Hopf link reconnections at $T=0K,1.9K$ and $2.1K$ (black, blue and red
respectively). The grey shaded areas
are the regions used to estimate the prefactors $A^{\pm}$. 
\emph{Inset:} Values of the separation prefactor $A^+$ and approach prefactors $A^-$. 
The dashed line corresponds to $A^+ = A^-$
\emph{(b)}: Comparison of all prefactors: Hopf links (\emph{HL}, circles), 
ring collisions (\emph{RC}, stars with yellow outline), GPE-data 
from Villois \etal \cite{villoisIrreversibleDynamicsVortex2020} (green diamonds) and experimental results from this study (orange triangles). 
The shaded areas associated with each colour represent the convex hull of errors 
for each temperature. Schematic rendering of initial conditions are included.}
\end{figure}

\paragraph*{Energy injection. ---}
The normal fluid impacts the dynamics of reconnecting superfluid vortices via the temperature dependent mutual friction coefficients. Conversely, the motion of superfluid vortices involved in the reconnection process influence, significantly, the dynamics of the normal fluid. Fig.~\ref{fig:energy-evol} indeed shows that the normal fluid energy, $E_n$,
suddenly increases at the reconnection time by an amount ($\approx 5\%$)
which is smaller but comparable to the continuous energy increase as vortex
lines approach each other. Indeed the
curvature $\zeta=|\s''|$ of the vortex line 
spikes at $t=t_0$ when
the reconnection cusp is created, and, in the first approximation \cite{galantucci-krstulovic-etal-2023},
the magnitude of the energy injected in the normal fluid per unit time $I$
is proportional to the strength of the mutual friction force $\mathbf{F}_{ns}$ which scales as
$|\mathbf{F}_{ns}(\s)|\propto|\dot{\s}-\v_n|\propto|\dot{\s}|\propto\zeta$. This sudden
transfer of energy
\cite{stasiakCrossComponentEnergyTransfer2024} from the superfluid
vortex configuration to the normal fluid is the origin of the 
small scale normal fluid enstrophy structures
which are visible in Fig.~\ref{fig:ring-coll-viz}.


%The prefactor ratio $A^+/A^-$ has been shown in recent literature 
%\cite{villoisIrreversibleDynamicsVortex2020,villoisUniversalNonuniversalAspects2017a,promentMatchingTheoryCharacterize2020} 
%to be of importance in describing fundamental properties of reconnections 
%using the Gross-Pitaevskii (GP) equation. Due to the \emph{ad-hoc} 
%nature of vortex reconnections in our helium model, it is not trivial 
%to derive a linear theory in the limit $\delta^{\pm}\rightarrow0$. 


\begin{figure}
	\centering
	\includegraphics*[width=0.45\textwidth]{energy-evolution.pdf}
	\caption{Normal fluid kinetic energy $E_n$ scaled by $E_n^0$ (the 
kinetic energy at $t=t_0$), plotted versus (dimensionless)
$\kappa (t-t_0)$ for the Hopf link reconnections. Black diamonds represent the simulations 
with minimum and maximum prefactor ratios $A^+/A^-$ at $T=1.9~\rm K$ and 
$T=2.1~\rm K$ respectively.}
	\label{fig:energy-evol}
\end{figure}

The total energy injected into the normal fluid by the reconnection,
$\Delta E_n$, which hereafter
we refer to as the energy jump, is defined as

\begin{equation}
	\Delta E_n = \max{\left[E_n(t>t_0)\right]} - E_n^0,
\end{equation} 

\noindent
where $E_n^0=E_n(t_0)$ is the normal fluid kinetic energy at $t=t_0$.
Normalized energy jumps are plotted in
Fig.~\ref{fig:energy-jumps} as a function of
the ratio $A^+/A^-$. Here, we observe that the larger $A^+/A^-$ is 
(i.e the more the reconnection is anti-parallel \cite{villoisIrreversibleDynamicsVortex2020}),
the smaller the normal fluid excitation is.

The emission of the sound pulse at the vortex reconnection 
\cite{leadbeaterSoundEmissionDue2001b} which is typical of the GPE model
is absent in our incompressible hydrodynamic approach. To model
this effect, the change of vortex length, $\Delta L$, created by the
vortex reconnection algorithm is always negative by construction
\cite{baggaleySensitivityVortexFilament2012a},
because, in the local induction approximation to the Biot-Savart law, 
the superfluid incompressible kinetic energy, $E_s$, is proportional to the vortex length,
$L$. Such procedure ensures that at $T=0~{\rm K}$ when a reconnection occurs
$\Delta E_s\propto \Delta L < 0$. %Therefore, at $T=0~{\rm K}$, in the absence of any dissipative 
%normal fluid, $\Delta E_s\propto \Delta L$. 
Consequentially, in the absence of any dissipative normal fluid,
the superfluid energy 
$E_s$ that would be transferred to the sound pulse, normalized with its value $E_s^0$ at 
reconnection, is $-\Delta L/L_0$., 
%should exhibit the same characteristic behavior as the normal fluid energy 
%which is injected, $\Delta E_n/E_n^0$. 
If these normalized energy jumps (black diamonds
in Fig.~\ref{fig:energy-jumps}) are compared to the results obtained with the 
compressible GPE \cite{villoisIrreversibleDynamicsVortex2020} (purple squares)
we find a good agreement, confirming that the model we employ, 
%despite the ad-hoc algorithm employed to take into account the reconnection events
is suitable for the investigation of the feature of single reconnection events.
%Indeed, the black diamonds
%in Fig.~\ref{fig:energy-jumps} shows that our $T=0~{\rm K}$ simulations are
%in good agreement with the GPE results (purple squares).



\paragraph*{Implications for turbulence. ---}
Our numerical results have implications for our understanding of 
quantum turbulence \cite{BSS2023}.
A fully developed turbulent tangle of vortices is
characterized by its vortex line density $\mathcal{L}$ (vortex length
per unit volume); the frequency 
of vortex reconnections per unit volume is 
${f=(\kappa/6\pi)\mathcal{L}^{5/2}\ln(\mathcal{L}^{-1/2}/a_0)}$
\cite{barenghi2004}. From Fig.~\ref{fig:energy-evol} we estimate the 
normal fluid reconnection relaxation time $\tau_n$ as the time 
after reconnection at which the normal fluid energy $E_n/E_0$ has decayed
to the pre-reconnection level: in our dimensionless units, $\kappa \tau_n \approx 0.25$. 
Using this timescale, we estimate that
the average vortex line density that is required to sustain the normal fluid 
in a perturbed state via frequent vortex reconnections is approximately
$\mathcal{L} \approx 10^7$ to $10^8\mathrm{m}^{-2}$. Experiments in $^4$He
\cite{schwarz1981,milliken1982,roche2008,roche2007,Babuin2014} and in $^3$He
\cite{bradley2006} can achieve vortex line densities much larger than this.


\begin{figure}
	\centering
	\includegraphics*[width=0.48\textwidth]{energy-jump.pdf}
	\caption{Normalized energy jumps $\Delta E_n/E_n^0$ for Hopf link 
reconnections.The solid black diamonds are the normalized change in line length $\Delta L/L_0$ in the
$T=0~\rm K$ case. Blue and red circle correspond to $T=1.9K$ and $T=2.1K$,respectively. The purple squares are from GPE simulations of
Villois et al. \cite{villoisIrreversibleDynamicsVortex2020}.}
	\label{fig:energy-jumps}
\end{figure}

\paragraph*{Conclusions.---} \red{We have conducted an experiment using active particle tracers} and a statistical numerical study
of vortex reconnections in a wide range of temperatures using a model of $^4$He which accounts for the coupled dynamics of superfluid and normal fluid components.
We have verified the scaling law of the minimum vortex distance 
$\delta^{\pm}=A^{\pm} (\kappa |t-t_0|)^{1/2}$ and found that the approach prefactor $A^-$ has a clear temperature dependence independent of the geometry \red{in both experiments and numerics}, in contrast to the
separation prefactor $A^+$. The prefactors are in good agreement
with GPE simulations \cite{villoisIrreversibleDynamicsVortex2020,allen2014} 
and classical fluid reconnections \cite{yaoSeparationScalingViscous2020}
revealing that vortex reconnections display a universal behaviour regardless of the nature
of the fluid (classical or quantum) and of temeperature. It is worth noting that the behaviour, as a function of $A^+/A^-$, of 
the energy injected in the normal fluid (at $T>0$) and of the energy transferred to sound (at $T=0$)
\cite{villoisIrreversibleDynamicsVortex2020,leadbeaterSoundEmissionDue2001b} is dissimilar: the former decreases as $A^+/A^-$
increases, the latter the opposite. This likely arises from the distinct physics governing the loss of superfluid 
incompressible kinetic energy: mutual friction at $T>0$, quantum pressure at $T=0$.
We have also found that a reconnection event suddenly injects an amount of energy 
into the normal fluid which is comparable to the energy transferred by friction
during the vortex approach. Applying these results to turbulence, we have
compared the decay time of the normal fluid structures created by a
reconnection to the frequency of reconnections in a vortex tangle, and argued
that, if the vortex line density is large enough, these punctuated
energy injections should sustain the normal fluid in a perturbed state.

\bibliography{references3}% Produces the bibliography via BibTeX.

% \input{main3.bbl}

\end{document}
%
% ****** End of file apssamp.tex ******


\end{document}
%
% ****** End of file apssamp.tex ******


\end{document}
%
% ****** End of file apssamp.tex ******


\end{document}
%
% ****** End of file apssamp.tex ******
