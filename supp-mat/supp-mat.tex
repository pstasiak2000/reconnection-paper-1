\documentclass[%
%  reprint,
% superscriptaddress,
% groupedaddress,
%unsortedaddress,
% runinaddress,
% frontmatterverbose, 
% preprint,
%preprintnumbers,
%nofootinbib,
%nobibnotes,
%bibnotes,
 amsmath,amssymb,
 aps,
prl,
%pra,
% prb,
% rmp,
%prstab,
% prstper,
%floatfix,
]{revtex4-2}


\usepackage{graphicx}% Include figure files
\usepackage{dcolumn}% Align table columns on decimal point
\usepackage{bm}% bold math
\usepackage{blindtext}
\usepackage{float}
\usepackage{caption}
\usepackage{cleveref}
\usepackage{subcaption}

\newcommand{\etal}{{\it et al.}~}
\newcommand{\bom}{\boldsymbol{\omega}}
\newcommand{\hel}{\textsuperscript{4}He }

\def \s{\mathbf{s}}
\def \v{\mathbf{v}}
\def \x{\mathbf{x}}
\def \r{\mathbf{r}}
\def \k{\mathbf{k}}

\def \cm{\mathrm{cm}}
\def \cms{\mathrm{cm/s}}
\def \sec{\mathrm{s}}
\def \K{\mathrm{K}}

\def\red{\textcolor{red}}


\begin{document}
\preprint{APS/123-QED}

\title{Supplementary material: Punctuated energy injection in superfluid helium-4 vortex reconnections}

\author{P. Z. Stasiak}
\author{A. Baggaley}
\author{C.F. Barenghi}
\affiliation{School of Mathematics, Statistics and Physics, Newcastle University, Newcastle upon Tyne, NE1 7RU, United Kingdom}

\author{G. Krstulovic}
\affiliation{Universit\'e C\^ote d'Azur, Observatoire de la C\^ote d'Azur, CNRS,Laboratoire Lagrangre, Boulevard de l'Observatoire CS 34229 - F 06304 NICE Cedex 4, France}

\author{L. Galantucci}
\affiliation{Istituto per le Applicazioni del Calcolo ``M. Picone" IAC CNR, Via dei Taurini 19, 00185 Roma, Italy}

\date{\today}% It is always \today, today,
             %  but any date may be explicitly specified

\maketitle

\section{Method}

Using Schwarz mesoscopic model \cite{schwarzThreedimensionalVortexDynamics1988a}, vortex lines can be described as space curves $\s(\xi,t)$ of infinitesimal thickness, with a single quantum of circulation $\kappa=h/m_4=9.97\times10^{-8}\text{m}^2/\text{s}$, where $h$ is Planck's constant, $m_4=6.65\times10^{-27}\text{kg}$ is the mass of one helium atom, $\xi$ is the natural parameterisation, arclength, and $t$ is time. These conditions are a good approximation, since the vortex core radius of superfluid \textsuperscript{4}He($a_0=10^{-10}\text{m}$) is much smaller than any of the length scale of interest in turbulent flows. The equation of motion is
\begin{equation}
	\dot{\s}(\xi,t) = \v_s + \frac{\beta}{1+\beta}\left[\v_{ns}\cdot \s'\right]\s' + \beta\s'\times\v_{ns}+\beta'\s'\times\left[\s'\times \v_{ns}\right],
\end{equation}
where $\dot{\s}=\partial\s/\partial t$, $\s'=\partial\s/\partial \xi$ is the unit tangent vector, $\v_{ns}=\v_n - \v_s$, $\v_n$ and $\v_s$ are the normal fluid and superfluid velocities at $\s$ and $\beta$,$\beta'$ are temperature and Reynolds number dependent mutual fricition coefficients \cite{galantucciNewSelfconsistentApproach2020b}. The superfluid velocity $\v_s$ at a point $\x$ is determined by the Biot-Savart law
\begin{equation}
	\v_s(\x,t) = \frac{\kappa}{4\pi}\oint_{\mathcal{T}}\frac{\s'(\xi,t)\times\left[\x-\s(\xi,t)\right]}{|\x-\s(\xi,t)|}d\xi,
\end{equation}
where $\mathcal{T}$ represents the entire vortex configuration.
There is currently a lack of a well-defined theory of vortex reconnections in superfluid helium, like for the Gross-Pitaevskii equation \cite{villoisIrreversibleDynamicsVortex2020,villoisUniversalNonuniversalAspects2017a,promentMatchingTheoryCharacterize2020a}. An \emph{ad hoc} vortex reconnection algorithm is employed to resolve the collisions of vortex lines \cite{baggaleySensitivityVortexFilament2012a}.

A \emph{two-way model} is crucial to understand the accurately interept the back-reaction effect of the normal fluid on the vortex line and vice-versa \cite{stasiakCrossComponentEnergyTransfer2024}. We self-consistently evolve the normal fluid $\v_n$ with a modified Navier-Stokes equation
\begin{equation}
	\frac{\partial \v_n}{\partial t} + (\v_n\cdot\nabla)\v_n = -\nabla\frac{p}{\rho} + \nu_n\nabla^2\v_n + \frac{F_{ns}}{\rho_n},
\end{equation}
\begin{equation}
	\mathbf{F}_{ns} = \oint_{\mathcal{T}}\mathbf{f}_{ns}\delta(\x-\x)d\xi, \quad \nabla\cdot\v_n=0,
\end{equation}
where $\rho=\rho_n + \rho_s$ is the total density, $\rho_n$ and $\rho_s$ are the normal fluid and superfluid densities, $p$ is the pressure, $\nu_n$ is the kinematic viscosity of the normal fluid and $\mathbf{f}_{ns}$ is the local friction per unit length \cite{galantucciCoupledNormalFluid2015a}
\begin{equation}
	\mathbf{f}_{ns} = -D\s'\times\left[\s'\times(\dot{\s}-\v_n)\right]-\rho_n\kappa\s'\times(\v_n-\dot{\s}), 
\end{equation}
where $D$ is a coefficient dependent on the vortex Reynolds number and intrinsic properties of the normal fluid.

The results in this Letter are reported in dimensionless units, where the characteristic length scale is $\tilde{\lambda}=D/D_0$, where $D^3=(1\times10^{-3}\text{m})^3$ is the dimensional cube size, $D_0^3=(2\pi)^3$ is the non-dimensional cubic computational domain. The time scale is given by $\tilde{\tau} = \tilde{\lambda}^2\nu_n^0/\nu_n$, where the non-dimensional viscosity $\nu_n^0$ resolves the small scales of the normal fluid. In these simulations, these quantities are $\tilde{\lambda}=1.59\times10^{-4}\text{cm}$, $\nu_n^0=0.16$ and $\tilde{\tau} = ???$.
We consider two distinct initial vortex geometries at $T=0K,1.9K$ and $2.1K$. The first is a Hopf link, two linked rings of radius $R\approx1$ with an offset in the $xy$-plane defined by parameters $\Delta l_x$ and $\Delta l_y$. The offsets are chosen so that $(\Delta l_x, \Delta l _y) \in \lbrace(0.125i,0.125j)|i,j=-3,\cdots,3 \rbrace$, a total of 49 reconnections for each temperature. The second geometry is a collision of vortex rings of radius $R\approx1$ in a tent-like configuration (see Fig. \ref{fig: ring-coll-viz}), making an angle $\alpha$ with the vertical. We take 12 realisations of $\alpha$, such that $\alpha\in\lbrace i\pi/13|i=1,\cdots,12\rbrace$. 

In both cases, normal fluid rings are initially superimposed to match the vortex lines, eliminating the transient phase of generating normal fluid structures. The Lagrangian discretisation of vortex lines is $\Delta\xi=0.025$ (a total of 668 discretiation points) with a timestep of $\Delta t_{VF}=1.25\times10^{-5}$. A total of $N=256^3$ Eulerian mesh points were used for the normal fluid, with a timestep of $\Delta t_{NS} = 40\Delta t_{VF}$.

\bibliography{references}% Produces the bibliography via BibTeX.

\end{document}