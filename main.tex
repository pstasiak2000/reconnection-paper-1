% ****** Start of file apssamp.tex ******
%
%   This file is part of the APS files in the REVTeX 4.2 distribution.
%   Version 4.2a of REVTeX, December 2014
%
%   Copyright (c) 2014 The American Physical Society.
%
%   See the REVTeX 4 README file for restrictions and more information.
%
% TeX'ing this file requires that you have AMS-LaTeX 2.0 installed
% as well as the rest of the prerequisites for REVTeX 4.2
%
% See the REVTeX 4 README file
% It also requires running BibTeX. The commands are as follows:
%
%  1)  latex apssamp.tex
%  2)  bibtex apssamp
%  3)  latex apssamp.tex
%  4)  latex apssamp.tex
%
\documentclass[%
 reprint,
% superscriptaddress,
%groupedaddress,
%unsortedaddress,
%runinaddress,
%frontmatterverbose, 
%preprint,
%preprintnumbers,
%nofootinbib,
%nobibnotes,
%bibnotes,
 amsmath,amssymb,
 aps,
 prl,
%pra,
% prb,
% rmp,
%prstab,
%prstper,
%floatfix,
]{revtex4-2}

\usepackage{graphicx}% Include figure files
\usepackage{dcolumn}% Align table columns on decimal point
\usepackage{bm}% bold math
\usepackage{blindtext}
\usepackage{float}
\usepackage{caption}
\usepackage{cleveref}
\usepackage{subcaption}
\usepackage{xcolor}
%\usepackage{hyperref}% add hypertext capabilities
%\usepackage[mathlines]{lineno}% Enable numbering of text and display math
%\linenumbers\relax % Commence numbering lines

%\usepackage[showframe,%Uncomment any one of the following lines to test 
%%scale=0.7, marginratio={1:1, 2:3}, ignoreall,% default settings
%%text={7in,10in},centering,
%%margin=1.5in,
%%total={6.5in,8.75in}, top=1.2in, left=0.9in, includefoot,
%%height=10in,a5paper,hmargin={3cm,0.8in},
%]{geometry}


\newcommand{\etal}{{\it et al.}~}
\newcommand{\bom}{\boldsymbol{\omega}}

% \newcommand{\sd}[3][\null]{\ensuremath{\dfrac{\d^{#1} #2}{\d #3^{#1}}}}%Standard derivative 
% \newcommand{\pd}[3][\null]{\ensuremath{\dfrac{\partial^{#1} #2}{\partial #3^{#1}}}}%Partial derivative
% \newcommand{\matd}[2][\null]{\ensuremath{\dfrac{\mathrm{D}^{#1} #2}{\mathrm{D} t^{#1}}}} %Material derivative

\def \s{\mathbf{s}}
\def \v{\mathbf{v}}
\def \x{\mathbf{x}}
\def \r{\mathbf{r}}
\def \k{\mathbf{k}}

\def \cm{\mathrm{cm}}
\def \cms{\mathrm{cm/s}}
\def \sec{\mathrm{s}}
\def \K{\mathrm{K}}

\def\red#1{\textcolor{red}{#1}}
\def\blue#1{\textcolor{blue}{#1}}
%%%%%%%%%%%%%%%%%%%%%%%%%%%%%%%%%



\begin{document}

\preprint{APS/123-QED}

\title{Superfluid vortex reconnections at non-zero temperatures}

\author{P. Z. Stasiak}
\author{A. Baggaley}
\author{C.F. Barenghi}
\affiliation{School of Mathematics, Statistics and Physics, Newcastle University, Newcastle upon Tyne, NE1 7RU, United Kingdom}

\author{G. Krstulovic}
\affiliation{Universit\'e C\^ote d'Azur, Observatoire de la C\^ote d'Azur, CNRS,Laboratoire Lagrangre, Boulevard de l'Observatoire CS 34229 - F 06304 NICE Cedex 4, France}

\author{L. Galantucci}
\affiliation{Istituto per le Applicazioni del Calcolo ``M. Picone" IAC CNR, Via dei Taurini 19, 00185 Roma, Italy}

\date{\today}% It is always \today, today,
             %  but any date may be explicitly specified

\begin{abstract}

\end{abstract}

%\keywords{Suggested keywords}%Use showkeys class option if keyword
   %display desired
\maketitle

\paragraph*{Introduction.---} Reconnections are the fundamental events that
change the topology of the field lines in fluids and plasmas during their
time evolution. Reconnections
thus determine important physical properties, 
such as mixing and inter-scale energy transfer in fluids, 
or solar flares and tokamak instabilities
in plasmas \cite{Chapman2010}. The nature of reconnections is more clearly
studied if the field lines are concentrated in well-separated
filamentary structures:
vortices in fluids and magnetic flux tubes in plasmas. In superfluid 
helium this concentration is extreme, providing an ideal context:
superfluid vorticity is confined to vortex lines of atomic thickness 
(approximately $a_0 \approx 10^{-10}~\rm m$); a further simplification 
is that, unlike what happens in ordinary fluids, the
circulation of a superfluid vortex  is constrained to the quantised value
$\kappa=h/m=9.97 \times 10^{-8}~\rm m^2/s$, 
where $m$ is the mass of one helium atom and $h$ is Planck's constant. 

It was in the superfluid context that it was theoretically and
experimentally recognized
\cite{nazarenko2003,bewley2008,paoletti2010,zuccherQuantumVortexReconnections2012a,villoisUniversalNonuniversalAspects2017a,galantucciCrossoverInteractionDriven2019a}
that reconnections share a universal property irrespective of the initial
condition: the minimum distance between reconnecting 
vortices, $\delta$, scales with time, $t$, according to the form

\begin{equation}
\label{eq:scaling}
	\delta^{\pm}(t) = A^{\pm} (\kappa|t-t_0|)^{1/2},
\end{equation} 

\noindent
where $t_0$ is the reconnection time, and the dimensionless
prefactors $A^-$ and $A^+$ refer respectively to before
($t<t_0$) and after ($t>t_0$) the reconnection. The same scaling law
was then found for reconnections in ordinary viscous fluids 
\cite{yaoSeparationScalingViscous2020}. In the case of a pure
superfluid at temperature $T=0$, theoretical work based on
the Gross-Pitaevskki equation (GPE) has shown that
$A^+>A^-$, that is, after the reconnection, vortex lines move away from 
each others faster than in the initial approach; this result has been
related to irreversibility \cite{villoisIrreversibleDynamicsVortex2020},
and is caused by a rarefaction pulse created immediately after the reconnection\cite{leadbeaterSoundEmissionDue2001b,zuccherQuantumVortexReconnections2012a} which
removes some of the kinetic energy of the vortex configuration.
This acoustic energy loss depends on
the ratio $A^+/A^-$, which in turns depends on the angle of collision
between the vortices \cite{villoisIrreversibleDynamicsVortex2020}. 

However most helium experiments
are performed at temperatures $T>1~\rm K$, a regime in which thermal 
excitations form a fluid called the {\it normal fluid} which provides a viscous
route to irreversibility. The aim of this Letter is to investigate
finite-temperature effects on vortex reconnections. 
To achieve this aim we need a more powerful
model than the GPE to account not only for the dynamics of the
superfluid vortices, but also for the dynamics of the normal fluid.
We shall show that at non-zero temperatures Eq.~\ref{eq:scaling} and the
relation $A^+>A-$ hold true, in agreement with experiments. Finally we shall
show that, when applied to a turbulent superfluid, a vortex
reconnection is a punctuated energy injection into the normal fluid.
When applied to turbulence, this last result implies that if the vortex line 
density (hence the frequency of reconnections) is large enough, vortex
reconnections can maintain the normal fluid in a perturbed state.




%%%%%%%%%%%%%%%%%%%%%%%%%%%%%%%%%%%%%%%%%%%%%%%%%%%%%%%%%%%
\begin{figure*}[t]
	\centering
	\begin{subfigure}[b]{0.24\textwidth}
		\centering
		\includegraphics*[width=\textwidth]{snap-1.png}
	\end{subfigure}
	\begin{subfigure}[b]{0.24\textwidth}
		\centering
		\includegraphics*[width=\textwidth]{snap-2.png}
	\end{subfigure}
    \begin{subfigure}[b]{0.24\textwidth}
		\centering
		\includegraphics*[width=\textwidth]{snap-3.png}
	\end{subfigure}
    \begin{subfigure}[b]{0.24\textwidth}
		\centering
		\includegraphics*[width=\textwidth]{snap-4.png}
	\end{subfigure}
    \hfill
    \vspace{0.5cm}
	\begin{subfigure}[b]{0.24\textwidth}
		\centering
		\includegraphics*[width=\textwidth]{snap-5.png}
	\end{subfigure}
	\begin{subfigure}[b]{0.24\textwidth}
		\centering
		\includegraphics*[width=\textwidth]{snap-6.png}
	\end{subfigure}
    \begin{subfigure}[b]{0.24\textwidth}
		\centering
		\includegraphics*[width=\textwidth]{snap-7.png}
	\end{subfigure}
    \begin{subfigure}[b]{0.24\textwidth}
		\centering
		\includegraphics*[width=\textwidth]{snap-8.png}
	\end{subfigure}



	\caption{3D rendering of vortex ring collisions, from a tent-like initial condition. The red tube represents a superfluid vortex, where the radius has been greatly exaggerated for visual purposes, and the blue volume rendering represents the scaled normal fluid enstrophy $\bom^2/\bom^2_{max}$. \emph{Top row:} Isometric view. \emph{Bottom row:} View of the $xy$-plane.}
	\label{fig: ring-coll-viz}
\end{figure*}

\paragraph*{Method.---}We follow the approach of Schwarz
\cite{schwarzThreedimensionalVortexDynamics1988a} which exploits
the vast separation of length scales between the vortex core $a_0$ and any
other relevant distance, in particular the average distance between vortices,
$\ell$, in the case of turbulence. Vortex lines are described
as space curves $\s(\xi,t)$ where $\xi$ is arclength. The equation of motion 
of the vortex lines is

\begin{equation}
	\dot{\s}(\xi,t) = \v_s + \frac{\beta}{(1+\beta)}\left[\v_{ns}\cdot \s'\right]\s' + \beta\s'\times\v_{ns}+\beta'\s'\times\left[\s'\times \v_{ns}\right],
\end{equation}

\noindent
where $\dot{\s}=\partial\s/\partial t$, $\s'=\partial\s/\partial \xi$ 
is the unit tangent vector, $\v_{ns}=\v_n - \v_s$, $\v_n$ and $\v_s$ 
are the normal fluid and superfluid velocities at $\s$,
and $\beta$, $\beta'$ are temperature and Reynolds number dependent 
mutual friction coefficients \cite{galantucciNewSelfconsistentApproach2020b}. 
The superfluid velocity $\v_s$ at a point $\x$ is determined by the 
Biot-Savart law

\begin{equation}
	\v_s(\x,t) = \frac{\kappa}{4\pi}\oint_{\mathcal{T}}\frac{\s'(\xi,t)\times\left[\x-\s(\xi,t)\right]}{|\x-\s(\xi,t)|}d\xi,
\end{equation}
where $\mathcal{T}$ denotes the entire vortex configuration. 
Superfluid vortices are coupled to a classical description of the 
incompressible ($\nabla\cdot\v_n=0$) normal fluid via the mutual 
friction force $\mathbf{F}_{ns}$

\begin{equation}
	\frac{\partial \v_n}{\partial t} + (\v_n\cdot\nabla)\v_n = 
        -\frac{1}{\rho} \nabla p + \nu_n\nabla^2\v_n + \frac{\mathbf{F}_{ns}}{\rho_n},
\end{equation}

\noindent
where $\rho=\rho_n + \rho_s$ is the total density, $\rho_n$ and 
$\rho_s$ are the normal fluid and superfluid densities, $p$ is the pressure, 
and $\nu_n$ is the kinematic viscosity of the normal fluid. 


\paragraph*{Initial configurations.---}
We consider two distinct initial vortex configurations
at $T=0K,1.9K$ and $2.1K$. The first is a Hopf link, which consists
of two linked rings of radius $R\approx1$ with an offset in the $xy$-plane 
defined by parameters $\Delta l_x$ and $\Delta l_y$. 
The offsets are chosen so that 
$(\Delta l_x, \Delta l _y) \in \lbrace(0.125i,0.125j)|i,j=-3,\cdots,3 \rbrace$,
providing us with a total of 49 reconnections for each temperature. 
The second configuration of two vortex rings of radius $R\approx1$ 
in a tent-like shape (see Fig. \ref{fig: ring-coll-viz}), 
making the angle $\alpha$ with the vertical direction. 
We take 12 realisations of $\alpha$, such that 
$\alpha\in\lbrace i\pi/13|i=1,\cdots,12\rbrace$. The supplementary material 
gives more details of our model and these configurations
\cite{SeeSupplementaryMaterials}. In all cases, normal fluid structures
in the form of rings are 
initially superimposed to match the vortex lines, eliminating the transient 
phase of generating these normal fluid structures.


\paragraph*{Results.---}
In the case of the Hopf the interactio between the two rings leads
to a reconnection.  We have performed 147 simulations 
(49 across 3 temperatures) as shown in Fig. \ref{fig: minimum-distance} and
verified Eq.~/ref{eq:scaling} for the minium distance $\delta$. 
The prefactors $A^{\pm}$ have been computed in the shaded region 
of the figure. In the pre-reconnection regime we observe
a clear segregation of the values of $A^-$ due to temperature. 
In stark contrast, there is no memory of the temperature in the
post-reconnection regime. 

HERE
Interestingly, the $A^{+}$ distribution does not drastically change when including finite temperature effects, suggesting a minor role of the normal fluid in the reconnection dynamics. Our results confirm the irreversibility of vortex reconnections in our finite temperature model, for which $A^{+}\geq A^{-}$. As shown in Fig. \ref{fig: prefactors}, the $T=0K$ calculation for helium is in good agreement with the Gross-Pitaevskii (GP) model where $A^-\sim0.4$-$0.6$, for both initial conditions. Recent investigation in the classical Navier-Stokes \cite{yaoSeparationScalingViscous2020} have displayed a clear $1/2$ power law scaling and pre-factor ratio $A^-\sim0.3$-$0.4$, which again shows good agreement with the results that we have presented here for finite temperature. 


\begin{figure}
	\centering
	\begin{subfigure}[b]{0.45\textwidth}
		\centering
		\includegraphics*[width=\textwidth]{min-delta.pdf}
		\caption{}
		\label{fig: minimum-distance}
	\end{subfigure}
	\begin{subfigure}[b]{0.45\textwidth}
		\centering
		\includegraphics*[width=\textwidth]{prefactors.pdf}
		\caption{}
		\label{fig: prefactors}
	\end{subfigure}
	\caption{\emph{(a)}: Time evolution of the minimum distance squared $\delta^2$ for the Hopf link initial conditions at $T=0K,1.9K$ and $2.1K$. The grey shaded area represens the vertical region used to estimate the prefactors $A^{\pm}$. \emph{Inset:} Values of the seperation prefactor $A^+$ and approach prefactors $A^-$. \emph{(b)}: Comparison of all prefactor values, \emph{HL}-Hopf link (circles), \emph{RC}-ring collision (stars with yellow outline), GPE-data from Gross-Pitaevskii simulations from Villois \etal \cite{villoisIrreversibleDynamicsVortex2020}. The shaded areas associated with each colour represent the convex hull of errors for each temperature. }
\end{figure}

The steep energy injection observed at reconnection time in the normal fluid kinetic energy $E_n$ is driven by the violent topological change in the vortex geometry. The coupling mutual fricition force $\mathbf{f}_{ns}(\s)$, in the first approximation $|\mathbf{f}_{ns}|\propto|\dot{\s}-\v_n|\propto\zeta$, where $\zeta=|\s''|$ is the curvature of the vortex line at $\s$. The curvature $\zeta$ spikes when the post-reconnection cusp is created, viciously stirring the normal fluid and generating excitations \cite{stasiakCrossComponentEnergyTransfer2024}, see Fig. \ref{fig: ring-coll-viz}. The prefactor ratio $A^+/A^-$ has been shown in recent literature \cite{villoisIrreversibleDynamicsVortex2020,villoisUniversalNonuniversalAspects2017a,promentMatchingTheoryCharacterize2020} to be of importance in describing fundamental properties of reconnections using the Gross-Pitaevskii (GP) equation. Due to the \emph{ad-hoc} nature of vortex recnnections in our helium model, it is not trivial to derive a linear theory in the limit $\delta^{\pm}\rightarrow0$. 


\begin{figure}[H]
	\centering
	\includegraphics*[width=0.45\textwidth]{energy-evolution.pdf}
	\caption{Total normal fluid kinetic energy $E_n$ scaled by the kinetic energy at reconnection time $E_n^0$. Black diamonds represent the simulations with minimum and maximum prefactor ratios $A^+/A^-$ at $T=1.9K$ and $T=2.1K$ respectively.}
\end{figure}

The total energy injected into the normal fluid by the reconnection $\Delta E_n$, which we refer to as energy jumps, is computed by 
\begin{equation}
	\Delta E_n = \max{\left[E_n(t>t_0)\right]} - E_n^0
\end{equation} 
where $E_n^0$ is the normal fluid kinetic energy at reconnection time $E_n^0=E_n(t_0)$. The normalised energy jumps are shown in Fig. \ref{fig: energy-jumps}, plotted against the prefactor ratio $A^+/A^-$. Here, we observe that a higher value of $A^+/A^-$, i.e a larger asymmetry in reconnection dynamics, in fact leads to smaller normal fluid excitations during the reconnection procedure. The emission of sound pulses is a common feature of superfluid vortex reconnections \cite{leadbeaterSoundEmissionDue2001b} which is restricted by our incompressible hydrodynamic model. In the local induction approximation, the superfluid kinetic energy is proportional to the length of filaments $E_s\propto L$. To compensate for this expulsion of superfluid kinetic energy, a portion of line length $\Delta L$ is removed during reconnection. At $T=0K$, the absence of a dissipative normal fluid removes any possible sinks of energy, and so ensures that any removal of energy $\Delta E_s\propto \Delta L$. Consequentially, as shown as black diamonds in Fig. \ref{fig: energy-jumps}, the superfluid energy $E_s$ that would be transferred to a sound pulse, $-\Delta L/L_0$, exhibits the same characteristic behaviour as the normal fluid energy injected $\Delta E_n/E_n^0$. Remarkably, the theoretical sound pulse energy $-\Delta L/L_0$ at $T=0K$ is in very good agreement with the energy pulse generated in the GP. 

\begin{figure}
	\centering
	\includegraphics*[width=0.48\textwidth]{energy-jump.pdf}
	\caption{The total energy jump $\Delta E_n$, the increase in normal fluid kinetic energy due to a superfluid vortex reconnection, for the Hopf links. The solid black diamond represents the change in line length $\Delta L$ in the $T=0K$ case, and the purple squares are from GP simulations from Villois \etal \cite{villoisIrreversibleDynamicsVortex2020}}
	\label{fig: energy-jumps}
\end{figure}

\blue{
\paragraph*{Closing remarks.---}
}

\bibliography{references}% Produces the bibliography via BibTeX.
\end{document}
%
% ****** End of file apssamp.tex ******
