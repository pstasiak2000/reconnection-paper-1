% ****** Start of file apssamp.tex ******
%
%   This file is part of the APS files in the REVTeX 4.2 distribution.
%   Version 4.2a of REVTeX, December 2014
%
%   Copyright (c) 2014 The American Physical Society.
%
%   See the REVTeX 4 README file for restrictions and more information.
%
% TeX'ing this file requires that you have AMS-LaTeX 2.0 installed
% as well as the rest of the prerequisites for REVTeX 4.2
%
% See the REVTeX 4 README file
% It also requires running BibTeX. The commands are as follows:
%
%  1)  latex apssamp.tex
%  2)  bibtex apssamp
%  3)  latex apssamp.tex
%  4)  latex apssamp.tex
%
\documentclass[%
 reprint,
% superscriptaddress,
%groupedaddress,
%unsortedaddress,
%runinaddress,
%frontmatterverbose, 
%preprint,
%preprintnumbers,
%nofootinbib,
%nobibnotes,
%bibnotes,
 amsmath,amssymb,
 aps,
 prl,
%pra,
% prb,
% rmp,
%prstab,
%prstper,
%floatfix,
]{revtex4-2}

\usepackage{graphicx}% Include figure files
\usepackage{dcolumn}% Align table columns on decimal point
\usepackage{bm}% bold math
\usepackage{blindtext}
\usepackage{float}
\usepackage{caption}
\usepackage{cleveref}
\usepackage{subcaption}
\usepackage{xcolor}
%\usepackage{hyperref}% add hypertext capabilities
%\usepackage[mathlines]{lineno}% Enable numbering of text and display math
%\linenumbers\relax % Commence numbering lines

%\usepackage[showframe,%Uncomment any one of the following lines to test 
%%scale=0.7, marginratio={1:1, 2:3}, ignoreall,% default settings
%%text={7in,10in},centering,
%%margin=1.5in,
%%total={6.5in,8.75in}, top=1.2in, left=0.9in, includefoot,
%%height=10in,a5paper,hmargin={3cm,0.8in},
%]{geometry}


\newcommand{\etal}{{\it et al.}~}
\newcommand{\bom}{\boldsymbol{\omega}}

% \newcommand{\sd}[3][\null]{\ensuremath{\dfrac{\d^{#1} #2}{\d #3^{#1}}}}%Standard derivative 
% \newcommand{\pd}[3][\null]{\ensuremath{\dfrac{\partial^{#1} #2}{\partial #3^{#1}}}}%Partial derivative
% \newcommand{\matd}[2][\null]{\ensuremath{\dfrac{\mathrm{D}^{#1} #2}{\mathrm{D} t^{#1}}}} %Material derivative

\def \s{\mathbf{s}}
\def \v{\mathbf{v}}
\def \x{\mathbf{x}}
\def \r{\mathbf{r}}
\def \k{\mathbf{k}}

\def \cm{\mathrm{cm}}
\def \cms{\mathrm{cm/s}}
\def \sec{\mathrm{s}}
\def \K{\mathrm{K}}

\def\red#1{\textcolor{red}{#1}}
\def\blue#1{\textcolor{blue}{#1}}
%%%%%%%%%%%%%%%%%%%%%%%%%%%%%%%%%



\begin{document}

\preprint{APS/123-QED}

\title{Punctuated energy injection in superfluid helium-4 vortex reconnections}

\author{P. Z. Stasiak}
\author{A. Baggaley}
\author{C.F. Barenghi}
\affiliation{School of Mathematics, Statistics and Physics, Newcastle University, Newcastle upon Tyne, NE1 7RU, United Kingdom}

\author{G. Krstulovic}
\affiliation{Universit\'e C\^ote d'Azur, Observatoire de la C\^ote d'Azur, CNRS,Laboratoire Lagrangre, Boulevard de l'Observatoire CS 34229 - F 06304 NICE Cedex 4, France}

\author{L. Galantucci}
\affiliation{Istituto per le Applicazioni del Calcolo ``M. Picone" IAC CNR, Via dei Taurini 19, 00185 Roma, Italy}

\date{\today}% It is always \today, today,
             %  but any date may be explicitly specified

\begin{abstract}
\blindtext
\end{abstract}

%\keywords{Suggested keywords}%Use showkeys class option if keyword
   %display desired
\maketitle

\blue{
\paragraph*{Introduction.---}\blindtext[4]
}

\begin{figure*}[t]
	\centering
	\begin{subfigure}[b]{0.24\textwidth}
		\centering
		\includegraphics*[width=\textwidth]{snap-1.png}
	\end{subfigure}
	\begin{subfigure}[b]{0.24\textwidth}
		\centering
		\includegraphics*[width=\textwidth]{snap-2.png}
	\end{subfigure}
    \begin{subfigure}[b]{0.24\textwidth}
		\centering
		\includegraphics*[width=\textwidth]{snap-3.png}
	\end{subfigure}
    \begin{subfigure}[b]{0.24\textwidth}
		\centering
		\includegraphics*[width=\textwidth]{snap-4.png}
	\end{subfigure}
    \hfill
    \vspace{0.5cm}
	\begin{subfigure}[b]{0.24\textwidth}
		\centering
		\includegraphics*[width=\textwidth]{snap-5.png}
	\end{subfigure}
	\begin{subfigure}[b]{0.24\textwidth}
		\centering
		\includegraphics*[width=\textwidth]{snap-6.png}
	\end{subfigure}
    \begin{subfigure}[b]{0.24\textwidth}
		\centering
		\includegraphics*[width=\textwidth]{snap-7.png}
	\end{subfigure}
    \begin{subfigure}[b]{0.24\textwidth}
		\centering
		\includegraphics*[width=\textwidth]{snap-8.png}
	\end{subfigure}



	\caption{3D rendering of vortex ring collisions, from a tent-like initial condition. The red tube represents a superfluid vortex, where the radius has been greatly exaggerated for visual purposes, and the blue volume rendering represents the scaled normal fluid enstrophy $\bom^2/\bom^2_{max}$. \emph{Top row:} Isometric view. \emph{Bottom row:} View of the $xy$-plane.}
	\label{fig: ring-coll-viz}
\end{figure*}

\paragraph*{Main results.---}The vast seperation of length scales between the vortex core $a_0$ and average distance between vortices in a tangle $\ell$ allows for vortices to be described as space curves $\s(\xi,t)$ with parameter $\xi$. The equation of motion of vortex lines follows from Schwarz \cite{schwarzThreedimensionalVortexDynamics1988a}
\begin{equation}
	\dot{\s}(\xi,t) = \v_s + \frac{\beta}{1+\beta}\left[\v_{ns}\cdot \s'\right]\s' + \beta\s'\times\v_{ns}+\beta'\s'\times\left[\s'\times \v_{ns}\right],
\end{equation}
here $\dot{\s}=\partial\s/\partial t$, $\s'=\partial\s/\partial \xi$ is the unit tangent vector, $\v_{ns}=\v_n - \v_s$, $\v_n$ and $\v_s$ are the normal fluid and superfluid velocities at $\s$ and $\beta$,$\beta'$ are temperature and Reynolds number dependent mutual fricition coefficients \cite{galantucciNewSelfconsistentApproach2020b}. The superfluid velocity $\v_s$ at a point $\x$ is determined by the Biot-Savart law
\begin{equation}
	\v_s(\x,t) = \frac{\kappa}{4\pi}\oint_{\mathcal{T}}\frac{\s'(\xi,t)\times\left[\x-\s(\xi,t)\right]}{|\x-\s(\xi,t)|}d\xi,
\end{equation}
where $\mathcal{T}$ represents the entire vortex configuration. Superfluid vortices are coupled to a classical description of the incompressible ($\nabla\cdot\v_n=0$) normal fluid via the mutual friciton force $\mathbf{F}_{ns}$
\begin{equation}
	\frac{\partial \v_n}{\partial t} + (\v_n\cdot\nabla)\v_n = -\nabla\frac{p}{\rho} + \nu_n\nabla^2\v_n + \frac{\mathbf{F}_{ns}}{\rho_n},
\end{equation}
where $\rho=\rho_n + \rho_s$ is the total density, $\rho_n$ and $\rho_s$ are the normal fluid and superfluid densities, $p$ is the pressure, $\nu_n$ is the kinematic viscosity of the normal fluid. We consider two distinct initial vortex geometries at $T=0K,1.9K$ and $2.1K$. The first is a Hopf link, two linked rings of radius $R\approx1$ with an offset in the $xy$-plane defined by parameters $\Delta l_x$ and $\Delta l_y$. The offsets are chosen so that $(\Delta l_x, \Delta l _y) \in \lbrace(0.125i,0.125j)|i,j=-3,\cdots,3 \rbrace$, a total of 49 reconnections for each temperature. The second geometry is a collision of vortex rings of radius $R\approx1$ in a tent-like configuration (see Fig. \ref{fig: ring-coll-viz}), making an angle $\alpha$ with the vertical. We take 12 realisations of $\alpha$, such that $\alpha\in\lbrace i\pi/13|i=1,\cdots,12\rbrace$. See the supplementary material for details on the model and dimensionality \cite{SeeSupplementaryMaterials}. In both cases, normal fluid rings are initially superimposed to match the vortex lines, eliminating the transient phase of generating normal fluid structures.


The linked structure of the Hopf link naturally decays due to self-induced propogation of vortex rings, leading to vortex reconnections. The smallest distance between two filaments around the reconnection event $\delta$, referred to in this Letter simply as `the minimum distance', has been shown to exhibit a power law behaviour
\begin{equation}
	\delta^{\pm}(t) = A^{\pm} (\kappa|t-t_0|)^{1/2},
\end{equation} 
where $\kappa$ is the quantum of circulation,$t_0$ is the reconnection time, $A^{\pm}$ are the dimensionless prefactors  where $+$ represents the \emph{seperation} of vortex filaments and $-$ the \emph{approach}. A total of 147 simulations (49 across 3 temperatures) for the Hopf links are shown in Fig. \ref{fig: minimum-distance}, where the prefactors which appear in the inset, are systematically computed using the shaded region for gradient estimation. The effect of viscous dissipation due to finite temperature effects is immediately clear in the pre-reconnection regime, a clear segregation of $\delta^-$ due to temperature. In stark contrast, any temperature-dependent segregation is lost after the reconnection event, bearing little to no resemblance to the order in the pre-reconnection regime. Interestingly, the $A^{+}$ distribution does not drastically change when including finite temperature effects, suggesting a minor role of the normal fluid in the reconnection dynamics. Our results confirm the irreversibility of vortex reconnections in our finite temperature model, for which $A^{+}\geq A^{-}$. As shown in Fig. \ref{fig: prefactors}, the $T=0K$ calculation for helium is in good agreement with the Gross-Pitaevskii (GP) model where $A^-\sim0.4$-$0.6$, for both initial conditions. Recent investigation in the classical Navier-Stokes \cite{yaoSeparationScalingViscous2020} have displayed a clear $1/2$ power law scaling and pre-factor ratio $A^-\sim0.3$-$0.4$, which again shows good agreement with the results that we have presented here for finite temperature. 


\begin{figure}
	\centering
	\begin{subfigure}[b]{0.45\textwidth}
		\centering
		\includegraphics*[width=\textwidth]{min-delta.pdf}
		\caption{}
		\label{fig: minimum-distance}
	\end{subfigure}
	\begin{subfigure}[b]{0.45\textwidth}
		\centering
		\includegraphics*[width=\textwidth]{prefactors.pdf}
		\caption{}
		\label{fig: prefactors}
	\end{subfigure}
	\caption{\emph{(a)}: Time evolution of the minimum distance squared $\delta^2$ for the Hopf link initial conditions at $T=0K,1.9K$ and $2.1K$. The grey shaded area represens the vertical region used to estimate the prefactors $A^{\pm}$. \emph{Inset:} Values of the seperation prefactor $A^+$ and approach prefactors $A^-$. \emph{(b)}: Comparison of all prefactor values, \emph{HL}-Hopf link (circles), \emph{RC}-ring collision (stars with yellow outline), GPE-data from Gross-Pitaevskii simulations from Villois \etal \cite{villoisIrreversibleDynamicsVortex2020}. The shaded areas associated with each colour represent the convex hull of errors for each temperature. }
\end{figure}

The steep energy injection observed at reconnection time in the normal fluid kinetic energy $E_n$ is driven by the violent topological change in the vortex geometry. The coupling mutual fricition force $\mathbf{f}_{ns}(\s)$, in the first approximation $|\mathbf{f}_{ns}|\propto|\dot{\s}-\v_n|\propto\zeta$, where $\zeta=|\s''|$ is the curvature of the vortex line at $\s$. The curvature $\zeta$ spikes when the post-reconnection cusp is created, viciously stirring the normal fluid and generating excitations \cite{stasiakCrossComponentEnergyTransfer2024a}, see Fig. \ref{fig: ring-coll-viz}. The prefactor ratio $A^+/A^-$ has been shown in recent literature \cite{villoisIrreversibleDynamicsVortex2020,villoisUniversalNonuniversalAspects2017a,promentMatchingTheoryCharacterize2020a} to be of importance in describing fundamental properties of reconnections using the Gross-Pitaevskii (GP) equation. Due to the \emph{ad-hoc} nature of vortex recnnections in our helium model, it is not trivial to derive a linear theory in the limit $\delta^{\pm}\rightarrow0$. 


\begin{figure}[H]
	\centering
	\includegraphics*[width=0.45\textwidth]{energy-evolution.pdf}
	\caption{Total normal fluid kinetic energy $E_n$ scaled by the kinetic energy at reconnection time $E_n^0$. Black diamonds represent the simulations with minimum and maximum prefactor ratios $A^+/A^-$ at $T=1.9K$ and $T=2.1K$ respectively.}
\end{figure}

The total energy injected into the normal fluid by the reconnection $\Delta E_n$, which we refer to as energy jumps, is computed by 
\begin{equation}
	\Delta E_n = \max{\left[E_n(t>t_0)\right]} - E_n^0
\end{equation} 
where $E_n^0$ is the normal fluid kinetic energy at reconnection time $E_n^0=E_n(t_0)$. The normalised energy jumps are shown in Fig. \ref{fig: energy-jumps}, plotted against the prefactor ratio $A^+/A^-$. Here, we observe that a higher value of $A^+/A^-$, i.e a larger asymmetry in reconnection dynamics, in fact leads to smaller normal fluid excitations during the reconnection procedure. The emission of sound pulses is a common feature of superfluid vortex reconnections \cite{leadbeaterSoundEmissionDue2001b} which is restricted by our incompressible hydrodynamic model. In the local induction approximation, the superfluid kinetic energy is proportional to the length of filaments $E_s\propto L$. To compensate for this expulsion of superfluid kinetic energy, a portion of line length $\Delta L$ is removed during reconnection. At $T=0K$, the absence of a dissipative normal fluid removes any possible sinks of energy, and so ensures that any removal of energy $\Delta E_s\propto \Delta L$. Consequentially, as shown as black diamonds in Fig. \ref{fig: energy-jumps}, the superfluid energy $E_s$ that would be transferred to a sound pulse, $-\Delta L/L_0$, exhibits the same characteristic behaviour as the normal fluid energy injected $\Delta E_n/E_n^0$. Remarkably, the theoretical sound pulse energy $-\Delta L/L_0$ at $T=0K$ is in very good agreement with the energy pulse generated in the GP. 

\begin{figure}
	\centering
	\includegraphics*[width=0.48\textwidth]{energy-jump.pdf}
	\caption{The total energy jump $\Delta E_n$, the increase in normal fluid kinetic energy due to a superfluid vortex reconnection, for the Hopf links. The solid black diamond represents the change in line length $\Delta L$ in the $T=0K$ case, and the purple squares are from GP simulations from Villois \etal \cite{villoisIrreversibleDynamicsVortex2020}}
	\label{fig: energy-jumps}
\end{figure}

\blue{
\paragraph*{Closing remarks.---}
\blindtext
}

\bibliography{references}% Produces the bibliography via BibTeX.
\end{document}
%
% ****** End of file apssamp.tex ******