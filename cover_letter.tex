\documentclass[a4paper,10pt]{letter}
\usepackage[margin=1in]{geometry}
\usepackage{lipsum}

\begin{document}

% Sender Information
\begin{flushright}
    \begin{tabular}{l}
        \textbf{Peter Stasiak} \\
        Newcastle Univeristy \\
        Newcastle  upon Tyne \\
        peter.stasiak@ncl.ac.uk \\
        \today
    \end{tabular}
\end{flushright}

\vspace{1cm}

% Recipient Information
Add recipients address\\


\vspace{1cm}

\textbf{Dear Editor,}

\vspace{0.5cm}

Dear Editor,

We are pleased to submit our manuscript entitled \textbf{``Superfluid Vortex Reconnections at Non-Zero Temperatures''} for consideration for publication in \textit{Physical Review Letters}. This paper presents experimental and numerical insights into the dynamics of vortex reconnections in superfluid helium at zero and finite temperatures, a study with significant implications for understanding of reconnection processes, energy dissipation and turbulence in quantum fluids.

For the first time we combine experiments with numerical simulations using both the Gross-Pitaevskii equation and the Vortex Filament Method. Our work investigates the universality of scaling of the minimum distance between reconnecting vortices, $\delta_{\pm} \sim |t - t_0|^{1/2}$, where $A_{-}$ and $A_{+}$ are the approach and separation prefactors associated with pre- and post-reconnection dynamics across a range of temperatures. 

We demonstrate that each reconnection event in a superfluid system injects energy into the normal fluid, a process with direct implications for sustaining turbulence in quantum fluids. By calculating the reconnection frequency and examining energy dissipation, we show that if the vortex line density is sufficiently high, these energy injections can maintain the normal fluid in a dynamically perturbed state.

We believe that our findings will be of interest to readers of \textit{Physical Review Letters}, as they provide a deeper understanding of superfluid dynamics and highlight universal aspects of reconnection behavior across fluid systems. Thank you for considering this manuscript; we look forward to your response.
\vspace{0.5cm}

\textbf{On behalf of the authors,}

\vspace{1cm}

\textbf{Peter Stasiak}

\end{document}
